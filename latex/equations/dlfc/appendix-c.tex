\documentclass{article}

% Document Layout and Fonts
\usepackage[margin=0.9in]{geometry}    % Page margins
\usepackage{fontspec}                  % For custom fonts (LuaLaTeX feature)
\usepackage{tgpagella}
\usepackage{mathpazo}
\setmainfont{EB Garamond}              % Main font (EB Garamond)
\usepackage{microtype}                 % Improves text appearance
\usepackage{titlesec}                  % Customize section title fonts

% Right-align section headings
\titleformat{\section}
  {\normalfont\large\scshape\raggedright}  % Right-align and small caps
  {}{0em}{}[]

% Right-align subsection headings and add a line below
\titleformat{\subsection}
  {\normalfont\normalsize\raggedleft}     % Right-align subsections
  {}{0em}{\titlerule[0.5pt]}              % Horizontal line below

% Right-align and italicize subsubsections
\titleformat{\subsubsection}
  {\normalfont\normalsize\itshape\raggedleft} % Right-align and italicize subsubsections
  {}{0em}{}[]

% Math and Science Packages
\usepackage{amsmath, amsfonts, amssymb, mathtools, amsthm, dsfont}

% Math commands and operators
\newcommand{\minus}{\scalebox{0.8}{\(-\)}}
\newcommand{\plus}{\scalebox{0.6}{\(+\)}}
\DeclareMathOperator{\sech}{sech}
\DeclareMathOperator{\sgn}{sgn}
\DeclareMathOperator{\tr}{Tr}
\newcommand{\diff}{\mathop{}\!\mathrm{d}}    % Differential d

% Definitions, theorems, corollaries, and friends
\usepackage[english]{babel}
\usepackage[hidelinks]{hyperref}
\newtheorem{axiom}{Axiom}
\newtheorem{postulate}{Postulate}
\newtheorem{definition}{Definition}
\newtheorem{lemma}{Lemma}
\newtheorem{theorem}{Theorem}
\newtheorem{corollary}{Corollary}
\newtheorem*{remark}{Remark}

\title{\LARGE\scshape\MakeUppercase{Appendix C. Lagrange Multipliers}}
\author{\textit{Bishop and Bishop}}
\date{}

\begin{document}

\maketitle

\begin{align*}
g\left(x_{1}, x_{2}\right)=0 
\tag{C.1}
\end{align*}

\begin{align*}
g(\mathbf{x}+\boldsymbol{\epsilon}) \simeq g(\mathbf{x})+\boldsymbol{\epsilon}^{\mathrm{T}} \nabla g(\mathbf{x}) 
\tag{C.2}
\end{align*}

\begin{align*}
\nabla f+\lambda \nabla g=0 
\tag{C.3}
\end{align*}

\begin{align*}
L(\mathbf{x}, \lambda) \equiv f(\mathbf{x})+\lambda g(\mathbf{x}) 
\tag{C.4}
\end{align*}

\begin{align*}
L(\mathbf{x}, \lambda)=1-x_{1}^{2}-x_{2}^{2}+\lambda\left(x_{1}+x_{2}-1\right) 
\tag{C.5}
\end{align*}

\begin{align*}
\begin{aligned}
-2 x_{1}+\lambda & =0  \tag{C.6}\\
-2 x_{2}+\lambda & =0  \tag{C.7}\\
x_{1}+x_{2}-1 & =0 
\end{aligned}
\tag{C.8}
\end{align*}

\begin{align*}
\begin{aligned}
g(\mathbf{x}) & \geqslant 0  \tag{C.9}\\
\lambda & \geqslant 0  \tag{C.10}\\
\lambda g(\mathbf{x}) & =0
\end{aligned}
\tag{C.11}
\end{align*}

\begin{align*}
L\left(\mathbf{x},\left\{\lambda_{j}\right\},\left\{\mu_{k}\right\}\right)=f(\mathbf{x})+\sum_{j=1}^{J} \lambda_{j} g_{j}(\mathbf{x})+\sum_{k=1}^{K} \mu_{k} h_{k}(\mathbf{x}) 
\tag{C.12}
\end{align*}

\end{document}
