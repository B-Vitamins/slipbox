\documentclass{article}

% Document Layout and Fonts
\usepackage[margin=0.9in]{geometry}    % Page margins
\usepackage{fontspec}                  % For custom fonts (LuaLaTeX feature)
\usepackage{tgpagella}
\usepackage{mathpazo}
\setmainfont{EB Garamond}              % Main font (EB Garamond)
\usepackage{microtype}                 % Improves text appearance
\usepackage{titlesec}                  % Customize section title fonts

% Right-align section headings
\titleformat{\section}
  {\normalfont\large\scshape\raggedright}  % Right-align and small caps
  {}{0em}{}[]

% Right-align subsection headings and add a line below
\titleformat{\subsection}
  {\normalfont\normalsize\raggedleft}     % Right-align subsections
  {}{0em}{\titlerule[0.5pt]}              % Horizontal line below

% Right-align and italicize subsubsections
\titleformat{\subsubsection}
  {\normalfont\normalsize\itshape\raggedleft} % Right-align and italicize subsubsections
  {}{0em}{}[]

% Math and Science Packages
\usepackage{amsmath, amsfonts, amssymb, mathtools, amsthm, dsfont}

% Math commands and operators
\newcommand{\minus}{\scalebox{0.8}{\(-\)}}
\newcommand{\plus}{\scalebox{0.6}{\(+\)}}
\DeclareMathOperator{\sech}{sech}
\DeclareMathOperator{\sgn}{sgn}
\DeclareMathOperator{\tr}{Tr}
\newcommand{\diff}{\mathop{}\!\mathrm{d}}    % Differential d

% Definitions, theorems, corollaries, and friends
\usepackage[english]{babel}
\usepackage[hidelinks]{hyperref}
\newtheorem{axiom}{Axiom}
\newtheorem{postulate}{Postulate}
\newtheorem{definition}{Definition}
\newtheorem{lemma}{Lemma}
\newtheorem{theorem}{Theorem}
\newtheorem{corollary}{Corollary}
\newtheorem*{remark}{Remark}

% Custom headers for the document
\title{\LARGE\scshape\MakeUppercase{Chapter 9: Optimization problems}}
\author{\textit{Hidetoshi Nishimori}}
\date{}

% Start of the document
\begin{document}

\maketitle
\section{Combinatorial optimization and statistical mechanics}

No equations.

\section{Number Partitioning Problem}

\subsection{Definition}

\begin{align*}
\tilde{E} = \left| \sum_{i \in B} a_i - \sum_{i \in A \backslash B} a_i \right| \tag{9.1}
\end{align*}

\subsection{Subset Sum}

\begin{align*}
H = \sum_{i=1}^{N} a_i n_i \tag{9.2}
\end{align*}

\begin{align*}
\tilde{H} = \left| \sum_{i=1}^{N} a_i S_i \right| = \left| 2H - \sum_{i=1}^{N} a_i \right| \tag{9.3}
\end{align*}

\begin{align*}
\tilde{C}(\tilde{E}) = 
\begin{cases}
    C\left( \frac{\tilde{E} + \sum_{i} a_i}{2} \right) + C\left( \frac{-\tilde{E} + \sum_{i} a_i}{2} \right) & \text{if } \tilde{E} \neq 0 \\
    C\left( \frac{\sum_{i} a_i}{2} \right) & \text{if } \tilde{E} = 0
\end{cases} \tag{9.4}
\end{align*}

\subsection{Number of Configurations for Subset Sum}

\begin{align*}
Z=\sum_{\left\{n_{i}\right\}} \mathrm{e}^{-\beta H}=\prod_{i=1}^{N}\left(1+\mathrm{e}^{-\beta a_{i}}\right) 
\tag{9.5}
\end{align*}

\begin{align*}
Z = \sum_{E=0}^{E_{\text{max}}} C(E) w^E \tag{9.6}
\end{align*}

\begin{align*}
C(E) = \frac{1}{2\pi i} \oint \frac{dw}{w^{E+1}} Z = \frac{1}{2\pi i} \oint \frac{e^{\log Z}}{w^{E+1}} dw \tag{9.7}
\end{align*}

\begin{align*}
E = \sum_{i=1}^{N} \frac{a_i}{1 + e^{\beta a_i}} \tag{9.8}
\end{align*}

\begin{align*}
C(E_0) = \frac{1}{2\pi} \int_{-\pi}^{\pi} d\theta \, \exp \left \{\log Z + \beta_0 E_0 - i E_0 \theta \right\} \tag{9.9}
\end{align*}

\begin{align*}
C(E_0) = \exp \left( \log Z(\beta_0) + \beta_0 E_0 \right) \cdot \frac{1}{2\pi} \int_{-\pi}^{\pi} d\theta \, \exp \left( -\frac{\theta^2}{2} \frac{\partial^2}{\partial \beta^2} \log Z (\beta_0) \right) \tag{9.10}
\end{align*}

\begin{align*}
C(E_0) = \frac{e^{\beta_0 E_0} \prod_{i} \left( 1 + e^{-\beta_0 a_i} \right)}{\sqrt{2\pi \sum_{i} a_i^2 / \left( 1 + e^{\beta_0 a_i} \right) \left( 1 + e^{-\beta_0 a_i} \right)}} \tag{9.11}
\end{align*}

\begin{align*}
E_0 = \sum_{i=1}^{N} \frac{a_i}{1 + e^{\beta_0 a_i}} \tag{9.12}
\end{align*}

\subsection{Number Partitioning Problem}

\begin{align*}
\tilde{C}(\tilde{E} = 0) = \frac{2^{N+1}}{\sqrt{2\pi \sum_{i} a_i^2}} \tag{9.13}
\end{align*}

\begin{align*}
\tilde{C}(\tilde{E}) = \binom{N}{(N + \tilde{E}) / 2} \tag{9.14}
\end{align*}

\clearpage

\section{Graph Partitioning Problem}

\subsection{Definition}

No equations

\subsection{Cost Function}

\begin{align*}
H &= -\sum_{i<j} J_{ij} S_i S_j = -\frac{J_{ij}}{2} \left( \sum_{i \in V_1, j \in V_1} + \sum_{i \in V_2, j \in V_2} + \sum_{i \in V_1, j \in V_2} + \sum_{i \in V_2, j \in V_1} \right) + \left( \sum_{i \in V_1, j \in V_2} + \sum_{i \in V_2, j \in V_1} \right) J_{ij} \\
& \qquad \qquad = -\frac{J}{2} \cdot \frac{2 N (N-1) p}{2} + 2 f(p) J
\tag{9.15}
\end{align*}

\begin{align*}
f(p) = \frac{H}{2J} + \frac{1}{4} N(N-1)p \tag{9.16}
\end{align*}

\begin{align*}
\sum_{i=1}^{N} S_i = 0 \tag{9.17}
\end{align*}

\subsection{Replica Expression}

\begin{align*}
\left[ Z^n \right] = (1-p)^{N(N-1)/2} \operatorname{tr} \prod_{i<j} \left\{ 1 + p_0 \exp \left( \beta J \sum_{\alpha=1}^{n} S_i^\alpha S_j^\alpha \right) \right\} \tag{9.18}
\end{align*}

\begin{align*}
\left[ Z^n \right] & = (1-p)^{N(N-1)/2} \exp \left\{ \frac{N(N-1)}{2} \log (1 + p_0) - \frac{N}{2} \left( \beta J c_1 n + \beta^2 J^2 c_2 n^2 \right) \right\} \\
&\quad \cdot \operatorname{tr} \exp \left\{ \frac{(\beta J)^2}{2} c_2 \sum_{\alpha,\beta} \left( \sum_i S_i^\alpha S_i^\beta \right)^2 + \mathcal{O}\left( \beta^3 J^3 \sum_{i<j} \left( \sum_{\alpha} S_i^\alpha S_j^\alpha \right)^3 \right) \right\}
\tag{9.19}
\end{align*}

\begin{align*}
c_j = \frac{1}{j!} \sum_{l=1}^{\infty} \frac{(-1)^{l-1}}{l} p_0^l l^j \tag{9.20}
\end{align*}

\begin{align*}
\sum_{l=1}^{\infty} \frac{(-1)^{l-1}}{l} p_{0}^{l} \cdot l \cdot \frac{1}{2} \sum_{\alpha}\left\{\left(\sum_{i} S_{i}^{\alpha}\right)^{2}-N\right\}=-\frac{N n}{2} c_{1}
\tag{9.21}
\end{align*}

\begin{align*}
\sum_{l=1}^{\infty} \frac{(-1)^{l-1}}{l} p_{0}^{l} & \sum_{i<j}\left(\sum_{\alpha} S_{i}^{\alpha} S_{j}^{\alpha}\right)^{2}\left(\binom{l}{2}+\frac{l}{2!}\right) \\
& = \sum_{l=1}^{\infty} \frac{(-1)^{l-1}}{l} p_{0}^{l} \cdot \frac{l^{2}}{2} \sum_{\alpha, \beta}\left\{\left(\sum_{i} S_{i}^{\alpha} S_{i}^{\beta}\right)^{2}-N\right\} = \frac{c_{2}}{2} \sum_{\alpha, \beta}\left(\sum_{i} S_{i}^{\alpha} S_{i}^{\beta}\right)^{2}-\frac{N n^{2}}{2} c_{2}
\tag{9.22}
\end{align*}

\subsection{Minimum of the Cost Function}

\begin{align*}
f(p) &= \frac{N^2}{4} p + \frac{1}{2J} E_g \\
E_g &= \lim_{\beta \to \infty} \lim_{n \to 0} \left( -\frac{1}{n \beta} \right) \left[ \operatorname{Tr} \exp \left\{ \frac{(\beta J)^2}{2} c_2 \sum_{\alpha,\beta} \left( \sum_i S_i^\alpha S_i^\beta \right)^2 + \mathcal{O} \left( \beta^3 J^3 \sum_{i<j} \left( \sum_\alpha S_i^\alpha S_j^\alpha \right)^3 \right) \right\} - 1 \right]
\tag{9.23}
\end{align*}

\begin{align*}
f(p)=\frac{N^{2}}{4} p+\frac{\sqrt{N}}{2 \tilde{J}} U_{0} \sqrt{c_{2}} \tilde{J} N=\frac{N^{2}}{4} p+\frac{1}{2} U_{0} N^{3 / 2} \sqrt{p(1-p)}
\tag{9.24}
\end{align*}

\clearpage

\section{Knapsack Problem}

\subsection{Knapsack Problem and Linear Programming}

\begin{align*}
U = \sum_{j=1}^{N} c_j \cdot \frac{S_j + 1}{2}, \quad Y = \sum_{j=1}^{N} a_j \cdot \frac{S_j + 1}{2} \leq b \tag{9.25}
\end{align*}

\begin{align*}
Y_k = \frac{1}{2} \sum_{j} a_{kj} \left( S_j + 1 \right) \leq b_k, \quad (k=1, \ldots, K) \tag{9.26}
\end{align*}

\begin{align*}
a_{kj} = \frac{1}{2} + \xi_{kj}, \quad P(\xi_{kj}) = \frac{1}{\sqrt{2\pi} \sigma} \exp \left( -\frac{\xi_{kj}^2}{2 \sigma^2} \right) \tag{9.27}
\end{align*}

\begin{align*}
Y_k - b = \frac{1}{2} \sum_j (1 + S_j) \xi_{kj} + \frac{1}{4} \sum_j S_j + \frac{N}{4} - b \leq 0 \tag{9.28}
\end{align*}

\subsection{Relaxation Method}

\begin{align*}
U = \frac{c N}{2} + \frac{c \sqrt{N}}{2} M \tag{9.29}
\end{align*}

\begin{align*}
Y_k = \frac{1}{2} \sum_j (1 + S_j) \xi_{kj} + \frac{\sqrt{N}}{4} M \leq 0, \quad M = \frac{1}{\sqrt{N}} \sum_j S_j \tag{9.30}
\end{align*}

\subsection{Replica Calculations}

\begin{align*}
\left[ V^n \right] = \left[ V_0^{-n} \int \prod_{\alpha, i} d S_i^\alpha \prod_{\alpha} \delta\left( \sum_j S_j^\alpha - \sqrt{N} M \right) \delta\left( \sum_j \left( S_j^\alpha \right)^2 - N \right) \cdot \prod_{\alpha, k} \Theta \left( -\frac{1}{\sqrt{N}} \sum_j (1 + S_j^\alpha) \xi_{kj} - \frac{M}{2} \right) \right] \tag{9.31}
\end{align*}

\begin{align*}
G = \alpha \int \mathrm{D} y \log L(q) + \frac{1}{2} \log (1 - q) + \frac{1}{2(1 - q)} \tag{9.32}
\end{align*}

\begin{align*}
L(q) = 2 \sqrt{\pi} \operatorname{Erfc} \left( \frac{M / 2 + y \sigma \sqrt{1 + q}}{\sqrt{2(1 - q)} \sigma} \right) \tag{9.33}
\end{align*}

\begin{align*}
\alpha(M_{\mathrm{opt}}) = \left\{ \frac{1}{4} \int_{-M_{\mathrm{opt}} / (2 \sqrt{2} \sigma)}^{\infty} \mathrm{D} y \left( \frac{M_{\mathrm{opt}}}{\sigma} + 2 \sqrt{2} y \right)^2 \right\}^{-1} \tag{9.34}
\end{align*}

\begin{align*}
\alpha \left\{ \int \mathrm{D} t \left( 1 - L_2 + L_1^2 \right) \right\}^2 = 1, \quad L_k = \frac{\int_{I_z} \mathrm{D} z ~ z^k}{\int_{I_z} \mathrm{D} z}, \quad \int_{I_z} = \int_{(M / 2\sigma + t \sqrt{1+q}) / \sqrt{1-q}}^{\infty} \tag{9.35}
\end{align*}

\clearpage

\section{Satisfiability Problem}

\subsection{Random Satisfiability Problem}

\begin{align*}
C_l = x_{i_1} \vee \bar{x}_{i_2} \vee \bar{x}_{i_3} \vee x_{i_4} \vee \cdots \vee x_{i_K} \tag{9.36}
\end{align*}

\begin{align*}
F \equiv C_1 \wedge C_2 \wedge \cdots \wedge C_M \tag{9.37}
\end{align*}

\begin{align*}
C_1 = x_1 \vee \bar{x}_2, \quad C_2 = x_1 \vee x_3, \quad C_3 = \bar{x}_2 \vee x_3 \tag{9.38}
\end{align*}

\subsection{Statistical-Mechanical Formulation}

\begin{align*}
E(\mathbf{S}) = \sum_{l=1}^{M} \delta \left( \sum_{i=1}^{N} C_{l i} S_i, -K \right) \tag{9.39}
\end{align*}

\begin{align*}
\left[ Z^n \right] = \operatorname{tr} \zeta_K(\mathbf{S})^M \tag{9.40}
\end{align*}

\begin{align*}
\zeta_K(\mathbf{S}) = \left[ \exp \left\{ -\frac{1}{T} \sum_{\alpha=1}^{n} \delta \left( \sum_{i=1}^{N} C_i S_i^{\alpha}, -K \right) \right\} \right] \tag{9.41}
\end{align*}

\begin{align*}
\delta \left( \sum_{i=1}^{N} C_i S_i^{\alpha}, -K \right) = \prod_{i=1; C_i \neq 0}^{N} \delta \left( S_i^{\alpha}, -C_i \right) \tag{9.42}
\end{align*}

\begin{align*}
\zeta_K(\mathbf{S}) = \frac{1}{2^K} \sum_{C_1 = \pm 1} \cdots \sum_{C_K = \pm 1} N^{-K} \sum_{i_1 = 1}^{N} \cdots \sum_{i_K = 1}^{N} \exp \left\{ -\frac{1}{T} \sum_{\alpha = 1}^{n} \prod_{k = 1}^{K} \delta(S_{i_k}^{\alpha}, -C_k) \right\}
\tag{9.43}
\end{align*}


\begin{align*}
N c(\boldsymbol{\sigma})=\sum_{i=1}^{N} \prod_{\alpha=1}^{n} \delta\left(S_{i}^{\alpha}, \sigma^{\alpha}\right) \tag{9.44}
\end{align*}

\begin{align*}
\zeta_{K}(\mathbf{S})= & \zeta_{K}(\{c\}) \\
& \equiv \frac{1}{2^{K}} \sum_{C_{1}= \pm 1} \cdots \sum_{C_{K}= \pm 1} \sum_{\boldsymbol{\sigma}_{1}} \cdots \sum_{\boldsymbol{\sigma}_{K}} c\left(-C_{1} \boldsymbol{\sigma}_{1}\right) \ldots c\left(-C_{K} \boldsymbol{\sigma}_{K}\right) \cdot \exp \left\{-\frac{1}{T} \sum_{\alpha=1}^{n} \prod_{k=1}^{K} \delta\left(\sigma_{k}^{\alpha}, 1\right)\right\}
\tag{9.45}
\end{align*}

\begin{align*}
c(\boldsymbol{\sigma})= \frac{1}{N} \sum_{i} \prod_{\alpha} \frac{1+S_{i}^{\alpha} \sigma_{i}^{\alpha}}{2} =\frac{1}{2^{n}}\left(1+\sum_{\alpha} Q^{\alpha} \sigma^{\alpha} + \sum_{\alpha<\beta} Q^{\alpha \beta} \sigma^{\alpha} \sigma^{\beta}+\sum_{\alpha<\beta<\gamma} Q^{\alpha \beta \gamma} \sigma^{\alpha} \sigma^{\beta} \sigma^{\gamma}+\ldots\right)
\tag{9.46}
\end{align*}

\begin{align*}
Q^{\alpha \beta \gamma \ldots}=\frac{1}{N} \sum_{i} S_{i}^{\alpha} S_{i}^{\beta} S_{i}^{\gamma} \cdots \tag{9.47}
\end{align*}

\begin{align*}
[Z^n] = \int \prod_{\boldsymbol{\sigma}} \mathrm{d}c(\boldsymbol{\sigma}) \, e^{-N E_0(\{c\})}
\cdot \operatorname{tr} \prod_{\boldsymbol{\sigma}} \delta \left( c(\boldsymbol{\sigma}) - N^{-1} \sum_{i=1}^{N} \prod_{\alpha=1}^{n} \delta(S_i^{\alpha}, \sigma^{\alpha}) \right)
\tag{9.48}
\end{align*}


\begin{align*}
E_0(\{c\}) = -\alpha \log \left\{ \sum_{\boldsymbol{\sigma}_1, \ldots, \boldsymbol{\sigma}_K} c(\boldsymbol{\sigma}_1) \cdots c(\boldsymbol{\sigma}_K) \prod_{\alpha=1}^{n} \left( 1 + \left( e^{-\beta} - 1 \right) \prod_{k=1}^{K} \delta \left( \sigma_k^{\alpha}, 1 \right) \right) \right\} \tag{9.49}
\end{align*}

\begin{align*}
\frac{N!}{\prod_{\boldsymbol{\sigma}}(N c(\boldsymbol{\sigma}))!}=\exp \left(-N \sum_{\boldsymbol{\sigma}} c(\boldsymbol{\sigma}) \log c(\boldsymbol{\sigma})\right) \tag{9.50}
\end{align*}

\begin{align*}
  -\frac{\beta F}{N}=-E_{0}(\{c\})-\sum_{\boldsymbol{\sigma}} c(\boldsymbol{\sigma}) \log c(\boldsymbol{\sigma}), \qquad \sum_{\boldsymbol{\sigma}} c(\boldsymbol{\sigma})=1 \tag{9.51}
\end{align*}

\subsection{Replica Symmetric Solution and Its Interpretation}

\begin{align*}
c(\boldsymbol{\sigma}) = \int_{-1}^{1} dm \, P(m) \prod_{\alpha=1}^{n} \frac{1 + m \sigma^{\alpha}}{2} \tag{9.52}
\end{align*}

\begin{align*}
P(m) = \frac{1}{2\pi(1 - m^2)} \int_{-\infty}^{\infty} du \, \cos \left( \frac{u}{2} \log \frac{1+m}{1-m} \right) \exp \left\{ -\alpha K + \alpha K \int_{-1}^{1} \prod_{k=1}^{K-1} dm_k \, P(m_k) \cos \left( \frac{u}{2} \log A_{K-1} \right) \right\} \tag{9.53}
\end{align*}

\begin{align*}
A_{K-1} = 1 + \left( e^{-\beta} - 1 \right) \prod_{k=1}^{K-1} \frac{1+m_k}{2} \tag{9.54}
\end{align*}

\begin{align*}
-\frac{\beta F}{N} &= \log 2+\alpha(1-K) \int_{-1}^{1} \prod_{k=1}^{K} \mathrm{~d} m_{k} P\left(m_{k}\right) \log A_{K} \\
&\qquad \qquad +\frac{\alpha K}{2} \int_{-1}^{1} \prod_{k=1}^{K-1} \mathrm{~d} m_{k} P\left(m_{k}\right) \log A_{K-1}-\frac{1}{2} \int_{-1}^{1} \mathrm{~d} m P(m) \log \left(1-m^{2}\right)
\tag{9.55}
\end{align*}

\begin{align*}
P(m)=\frac{1}{2 \pi\left(1-m^{2}\right)} \int_{-\infty}^{\infty} \mathrm{d} u \cos \left(\frac{u}{2} \log \frac{1+m}{1-m}\right) \exp \left(-\alpha+\alpha \cos (u \beta / 2)\right)
\tag{9.56}
\end{align*}

\begin{align*}
P(m)=\mathrm{e}^{-\alpha} \sum_{k=-\infty}^{\infty} I_{k}(\alpha) \delta\left(m-\tanh \frac{\beta k}{2}\right) \tag{9.57}
\end{align*}

\begin{align*}
P(m) = e^{-\alpha} I_0(\alpha) \delta(m) + \frac{1}{2} \left( 1 - e^{-\alpha} I_0(\alpha) \right) \left( \delta(m - 1) + \delta(m + 1) \right) \tag{9.58}
\end{align*}

\begin{align*}
-\frac{\beta F}{N}=\log 2-\frac{\alpha \beta}{2}+\mathrm{e}^{-\alpha} \sum_{k=-\infty}^{\infty} I_{k}(\alpha) \log \cosh \frac{\beta k}{2} \tag{9.59}
\end{align*}

\begin{align*}
\frac{E(\alpha)}{N}=\frac{\alpha}{2}-\frac{\alpha}{2} \mathrm{e}^{-\alpha}\left(I_{0}(\alpha)+I_{1}(\alpha)\right) \tag{9.60}
\end{align*}

\begin{align*}
\frac{S}{N} = e^{-\alpha} \left( I_0(\alpha) - 1 \right) \log 2 \tag{9.61}
\end{align*}

\clearpage

\section{Simulated annealing}

\subsection{Simulated annealing}

No equations.

\subsection{Annealing schedule and generalized transition probability}

No equations.

\subsection{Inhomogeneous Markov chain}

\begin{align*}
G(x, y ; t)= \begin{cases}P(x, y) A(x, y ; T(t)) & (x \neq y) \\ 1-\sum_{z(\neq x)} P(x, z) A(x, z ; T(t)) & (x=y)\end{cases}
\tag{9.62}
\end{align*}

\begin{align*}
P(x, y) = \begin{cases}>0 & \left(y \in \mathcal{S}_{x}\right) \\ =0 & \text { (otherwise) }\end{cases}
\tag{9.63}
\end{align*}

\begin{align*}
A(x, y ; T)=\min \{1, u(x, y ; T)\}, \qquad u(x, y ; T)=\left(1+(q-1) \frac{f(y)-f(x)}{T}\right)^{1 /(1-q)}
\tag{9.64}
\end{align*}

\begin{align*}
T(t)=\frac{b}{(t+2)^{c}} \quad(b, c>0, t=0,1,2, \ldots)
\tag{9.65}
\end{align*}

\begin{align*}
[G(t)]_{x, y}=G(x, y ; t)
\tag{9.66}
\end{align*}

\begin{align*}
p(s, t)=p_{0} G^{s, t} \equiv p_{0} G(s) G(s+1) \ldots G(t-1)
\tag{9.67}
\end{align*}

\begin{align*}
\alpha\left(G^{s, t}\right)=1-\min \left\{\sum_{z \in \mathcal{S}} \min \left\{G^{s, t}(x, z), G^{s, t}(y, z)\right\} \mid x, y \in \mathcal{S}\right\}
\tag{9.68}
\end{align*}

\begin{align*}
\forall s \geq 0: \lim _{t \rightarrow \infty} \sup\left\{\left\|p_{1}(s, t)-p_{2}(s, t)\right\| \mid p_{01}, p_{02} \in \mathcal{P}\right\}=0
\tag{9.69}
\end{align*}

\begin{align*}
p_{1}(s, t)=p_{01} G^{s, t}, \quad p_{2}(s, t)=p_{02} G^{s, t}
\tag{9.70}
\end{align*}

\begin{align*}
\left\|p_{1}-p_{2}\right\|=\sum_{x \in \mathcal{S}}\left|p_{1}(x)-p_{2}(x)\right|
\tag{9.71}
\end{align*}

\begin{align*}
\exists r \in \mathcal{P}, \forall s \geq 0: \lim _{t \rightarrow \infty} \sup\left\{\|p(s, t)-r\| \mid p_{0} \in \mathcal{P}\right\}=0
\tag{9.72}
\end{align*}

\begin{theorem}[Weak ergodicity]
An inhomogeneous Markov chain is weakly ergodic if and only if there exists a monotonically increasing sequence of integers
\begin{align*}
0 < t_{0} < t_{1} < \cdots < t_{i} < t_{i+1} < \cdots
\end{align*}
and the coefficient of ergodicity satisfies
\begin{align*}
\sum_{i=0}^{\infty} (1 - \alpha (G^{t_{i}, t_{i+1}})) = \infty
\tag{9.73}
\end{align*}
\end{theorem}

\begin{theorem}[Strong ergodicity]
An inhomogeneous Markov chain is strongly ergodic if:
\begin{enumerate}
    \item it is weakly ergodic,
    \item there exists a stationary state \(p_{t} = p_{t} G(t)\) at each \(t\), and
    \item the above \(p_{t}\) satisfies the condition
\begin{align*}
\sum_{t=0}^{\infty}\lvert p_{t}-p_{t+1} \rvert < \infty
\tag{9.74}
\end{align*}
\end{enumerate}
\end{theorem}

\subsection{Weak ergodicity}

\begin{lemma}[Lower bound to the transition probability]
The transition probability of the inhomogeneous Markov chain defined in (9.62) and (9.63) satisfies the following inequality. Off-diagonal elements of \(G\) for transitions between different states satisfy
\begin{align*}
P(x, y)>0 \Rightarrow \forall t \geq 0: G(x, y ; t) \geq w\left(1+\frac{(q-1) L}{T(t)}\right)^{1 /(1-q)} \tag{9.75}
\end{align*}
For diagonal elements we have
\begin{align*}
\forall x \in \mathcal{S} \backslash \mathcal{S}^{M}, \exists t_{1}>0, \forall t \geq t_{1}: G(x, x ; t) \geq w\left(1+\frac{(q-1) L}{T(t)}\right)^{1 /(1-q)} \tag{9.76}
\end{align*}
Here \(\mathcal{S}^{M}\) is the set of states to locally maximize the cost function
\begin{align*}
\mathcal{S}^{M}=\left\{x \mid x \in \mathcal{S}, \forall y \in \mathcal{S}_{x}: f(y) \leq f(x)\right\} \tag{9.77}
\end{align*}
\(L\) is the largest value of the change of the cost function by a single step
\begin{align*}
L=\max \{|f(x)-f(y)| \mid P(x, y)>0\}
\tag{9.78}
\end{align*}
and \(w\) is the minimum value of the non-vanishing generation probability
\begin{align*}
w=\min \{P(x, y) \mid P(x, y)>0, x, y \in \mathcal{S}\}. \tag{9.79}
\end{align*}
\end{lemma}

\begin{align*}
G(x, y ; t) & =P(x, y) A(x, y ; T(t)) \geq w \min \{1, u(x, y ; T(t))\} =w u(x, y ; T(t)) \geq w\left(1+\frac{(q-1) L}{T(t)}\right)^{1 /(1-q)}
\tag{9.80}
\end{align*}

\begin{align*}
G(x, y ; t) \geq w \min \{1, u(x, y ; T(t))\}=w \geq w\left(1+\frac{(q-1) L}{T(t)}\right)^{1 /(1-q)} \tag{9.81}
\end{align*}

\begin{align*}
\lim _{t \rightarrow \infty} u(x, y ; T(t))=0 \tag{9.82}
\end{align*}

\begin{align*}
\lim _{t \rightarrow \infty} \min \{1, u(x, y ; T(t))\}=0 \tag{9.83}
\end{align*}

\begin{align*}
\forall t \geq t_{1}: \min \{1, u(x, y ; T(t))\}<\epsilon \tag{9.84}
\end{align*}

\begin{align*}
 \sum_{z \in \mathcal{S}} P(x, z) A(x, z ; T(t)) &=\sum_{\left\{y_{+}\right\}} P\left(x, y_{+}\right) \min \left\{1, u\left(x, y_{+} ; T(t)\right)\right\}+\sum_{z \in \mathcal{S} \backslash\left\{y_{+}\right\}} P(x, z) \min \{1, u(x, z ; T(t))\} \\
&\qquad <\sum_{\left\{y_{+}\right\}} P\left(x, y_{+}\right) \epsilon+\sum_{z \in \mathcal{S} \backslash\left\{y_{+}\right\}} P(x, z)=-(1-\epsilon) \sum_{\left\{y_{+}\right\}} P\left(x, y_{+}\right)+1
\tag{9.85}
\end{align*}

\begin{align*}
G(x, x ; t) \geq(1-\epsilon) \sum_{\left\{y_{+}\right\}} P\left(x, y_{+}\right) \geq w\left(1+\frac{(q-1) L}{T(t)}\right)^{1 /(1-q)}
\tag{9.86}
\end{align*}

\begin{align*}
k(x)=\max \{d(x, y) \mid y \in \mathcal{S}\}
\tag{9.87}
\end{align*}

\begin{align*}
R=\min \left\{k(x) \mid x \in \mathcal{S} \backslash \mathcal{S}^{M}\right\}, \quad x^{*}=\arg \min \left\{k(x) \mid x \in \mathcal{S} \backslash \mathcal{S}^{M}\right\}
\tag{9.88}
\end{align*}

\begin{theorem}[Weak ergodicity by generalized transition probability]
The inhomogeneous Markov chain defined by the transition probabilities in (9.64) is weakly ergodic if \(0<c \leq (q-1) / R\).
\end{theorem}

\begin{align*}
G^{t-R, t}\left(x, x^{*}\right)=\sum_{x_{1}, \ldots, x_{R-1}} G\left(x, x_{1} ; t-R\right) G\left(x_{1}, x_{2} ; t-R+1\right) \ldots G\left(x_{R-1}, x^{*} ; t-1\right)
\tag{9.89}
\end{align*}

\begin{align*}
x \neq x_{1} \neq x_{2} \neq \cdots \neq x_{k}=x_{k+1}=\cdots=x_{R}=x^{*} \tag{9.90}
\end{align*}


\begin{align*}
G^{t-R, t}\left(x, x^{*}\right) & \geq G\left(x, x_{1} ; t-R\right) G\left(x_{1}, x_{2} ; t-R+1\right) \ldots G\left(x_{R-1}, x_{R} ; t-1\right) \\
& \geq \prod_{k=1}^{R} w\left(1+\frac{(q-1) L}{T(t-R+k-1)}\right)^{1 /(1-q)} \\
& \geq w^{R}\left(1+\frac{(q-1) L}{T(t-1)}\right)^{R /(1-q)}
\tag{9.91}
\end{align*}

\begin{align*}
\alpha\left(G^{t-R, t}\right) & =1-\min \left\{\sum_{z \in \mathcal{S}} \min \left\{G^{t-R, t}(x, z), G^{t-R, t}(y, z)\right\} \mid x, y \in \mathcal{S}\right\} \\
& \leq 1-\min \left\{\min \left\{G^{t-R, t}\left(x, x^{*}\right), G^{t-R, t}\left(y, x^{*}\right)\right\} \mid x, y \in \mathcal{S}\right\} \\
& \leq 1-w^{R}\left(1+\frac{(q-1) L}{T(t-1)}\right)^{R /(1-q)}
\tag{9.92}
\end{align*}

\begin{align*}
1-\alpha\left(G^{k R-R, k R}\right) & \geq w^{R}\left(1+\frac{(q-1) L(k R+1)^{c}}{b}\right)^{R /(1-q)} \\
& \geq w^{R}\left\{\frac{2(q-1) L R^{c}}{b}\left(k+\frac{1}{R}\right)^{c}\right\}^{R /(1-q)}
\tag{9.93}
\end{align*}

\begin{align*}
\sum_{k=0}^{\infty}\left(1-\alpha\left(G^{k R-R, k R}\right)\right)=\sum_{k=0}^{k_{0}-1}\left(1-\alpha\left(G^{k R-R, k R}\right)\right)+\sum_{k=k_{0}}^{\infty}\left(1-\alpha\left(G^{k R-R, k R}\right)\right) \tag{9.94}
\end{align*}

\subsection{Relaxation of the cost function}

\begin{theorem}[Strong ergodicity by conventional transition probability]
If we replace (9.64) and (9.65) by
\begin{align*}
u(x, y ; T) & =\exp \{-(f(y)-f(x)) / T\}  \tag{9.95}\\
T(t) & \geq \frac{R L}{\log (t+2)}
\tag{9.96}
\end{align*}
then this Markov chain is strongly ergodic. The probability distribution in the limit \(t \rightarrow \infty\) converges to the optimal distribution. Here \(R\) and \(L\) are defined as in (9.78) and (9.88) respectively.
\end{theorem}

\begin{align*}
t_{1} \approx \exp \left(\frac{k_{1} N}{q-1} \log \frac{b}{\delta}\right) \tag{9.97}
\end{align*}

\begin{align*}
t_{2} \approx \exp \left(\frac{k_{2} N}{\delta}\right) \tag{9.98}
\end{align*}

\begin{align*}
u_{1}(T=\delta) \approx\left(\frac{\delta}{(q-1) \Delta f}\right)^{1 /(q-1)} \tag{9.99}
\end{align*}

\begin{align*}
u_{2}(T=\delta) \approx \mathrm{e}^{-\Delta f / \delta} \tag{9.100}
\end{align*}

\clearpage

\section{Diffusion in one dimension}

\subsection{Diffusion and relaxation in one dimension}

\begin{align*}
\frac{\mathrm{d} P_{t}(i)}{\mathrm{d} t} &=\left(1+(q-1) \frac{B}{T}\right)^{\frac{1}{1-q}} P_{t}(i+1) +\left(1+(q-1) \frac{B+\Delta_{i-1}}{T}\right)^{\frac{1}{1-q}} P_{t}(i-1) \\
&\qquad \quad -\left(1+(q-1) \frac{B+\Delta_{i}}{T}\right)^{\frac{1}{1-q}} P_{t}(i) -\left(1+(q-1) \frac{B}{T}\right)^{\frac{1}{1-q}} P_{t}(i)
\tag{9.101}
\end{align*}

\begin{align*}
\gamma(T) =\frac{1}{T} \left(1+(q-1) \frac{B}{T}\right)^{\frac{q}{1-q}}, \qquad D(T) = \left(1+(q-1) \frac{B}{T}\right)^{\frac{1}{1-q}}
\tag{9.102}
\end{align*}

\begin{align*}
\frac{\mathrm{d} P_{t}(i)}{\mathrm{d} t} =D(T)\left\{P_{t}(i+1)-2 P_{t}(i)+P_{t}(i-1)\right\} +a \gamma(T)\left\{x P_{t}(i)+\frac{a}{2} P_{t}(i)-x P_{t}(i-1)+\frac{a}{2} P_{t}(i-1)\right\}
\tag{9.103}
\end{align*}

\begin{align*}
\frac{\partial P}{\partial t} &= \gamma(T) \frac{\partial}{\partial x}(x P) + D(T) \frac{\partial^{2} P}{\partial x^{2}}
\tag{9.104}
\end{align*}

\begin{align*}
y(t) &= \int \mathrm{d}x \, f(x) P(x,t)
\tag{9.105}
\end{align*}

\begin{align*}
\frac{\mathrm{d} y}{\mathrm{d} t} &= -2 \gamma(T) y + D(T)
\tag{9.106}
\end{align*}

\begin{align*}
T_{\mathrm{opt}} &= \frac{2 y B + (1-q) B^2}{2 y + B} = (1-q)B + 2q y + \mathcal{O}(y^2)
\tag{9.107}
\end{align*}

\begin{align*}
\frac{\mathrm{d} y}{\mathrm{d} t} &= -2 B^{\frac{1}{q-1}} \left(\frac{2q}{1-q}\right)^{\frac{q}{1-q}} y^{\frac{1}{1-q}}
\tag{9.108}
\end{align*}

\begin{align*}
y &= B^{\frac{1}{q}} \left(\frac{1-q}{2q}\right)^{\frac{1}{q}} t^{-\frac{1-q}{q}}
\tag{9.109}
\end{align*}

\begin{align*}
T_{\mathrm{opt}} &\approx (1-q) B + \mathrm{const} \cdot t^{-\frac{1-q}{q}}
\tag{9.110}
\end{align*}

\begin{align*}
y &\approx \frac{B^2}{4} t^{-1}, \quad T_{\mathrm{opt}} \approx \frac{B}{2} + \frac{B^2}{4} t^{-1}
\tag{9.111}
\end{align*}

\begin{align*}
y &\approx \frac{B}{\log t}, \quad T_{\mathrm{opt}} \approx \frac{B}{\log t}
\tag{9.112}
\end{align*}

\end{document}
