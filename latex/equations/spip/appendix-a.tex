\documentclass{article}

% Document Layout and Fonts
\usepackage[margin=0.9in]{geometry}    % Page margins
\usepackage{fontspec}                  % For custom fonts (LuaLaTeX feature)
\usepackage{tgpagella}
\usepackage{mathpazo}
\setmainfont{EB Garamond}              % Main font (EB Garamond)
\usepackage{microtype}                 % Improves text appearance
\usepackage{titlesec}                  % Customize section title fonts

% Right-align section headings
\titleformat{\section}
  {\normalfont\large\scshape\raggedright}  % Right-align and small caps
  {}{0em}{}[]

% Right-align subsection headings and add a line below
\titleformat{\subsection}
  {\normalfont\normalsize\raggedleft}     % Right-align subsections
  {}{0em}{\titlerule[0.5pt]}              % Horizontal line below

% Right-align and italicize subsubsections
\titleformat{\subsubsection}
  {\normalfont\normalsize\itshape\raggedleft} % Right-align and italicize subsubsections
  {}{0em}{}[]

% Math and Science Packages
\usepackage{amsmath, amsfonts, amssymb, mathtools, amsthm, dsfont}

% Math commands and operators
\newcommand{\minus}{\scalebox{0.8}{\(-\)}}
\newcommand{\plus}{\scalebox{0.6}{\(+\)}}
\DeclareMathOperator{\sech}{sech}
\DeclareMathOperator{\sgn}{sgn}
\DeclareMathOperator{\tr}{Tr}
\newcommand{\diff}{\mathop{}\!\mathrm{d}}    % Differential d

% Definitions, theorems, corollaries, and friends
\usepackage[english]{babel}
\usepackage[hidelinks]{hyperref}
\newtheorem{axiom}{Axiom}
\newtheorem{postulate}{Postulate}
\newtheorem{definition}{Definition}
\newtheorem{lemma}{Lemma}
\newtheorem{theorem}{Theorem}
\newtheorem{corollary}{Corollary}
\newtheorem*{remark}{Remark}

\title{\LARGE\scshape\MakeUppercase{Appendix A: Eigenvalues of the Hessian}}
\author{\textit{Hidetoshi Nishimori}}
\date{}

% Start of the document
\begin{document}

\maketitle

\section{Eigenvalue 1}

\begin{align*}
G \boldsymbol{\mu} &= \lambda \boldsymbol{\mu}, \quad \boldsymbol{\mu} = \binom{\left\{\epsilon^{\alpha}\right\}}{\left\{\eta^{\alpha \beta}\right\}} 
\tag{A.1}
\end{align*}

\begin{align*}
(A, B, \ldots, B, C, \ldots, C, D, \ldots, D)
\tag{A.2}
\end{align*}

\begin{align*}
A a + (n-1) B a + (n-1) C b + \frac{1}{2}(n-1)(n-2) D b &= \lambda_1 a 
\tag{A.3}
\end{align*}

\begin{align*}
2 C a + (n-2) D a + P b + 2(n-2) Q b + \frac{1}{2}(n-2)(n-3) R b &= \lambda_1 b 
\tag{A.4}
\end{align*}

\begin{align*}
\lambda_1 &= \frac{1}{2}\left(X \pm \sqrt{Y^2 + Z}\right)
\tag{A.5}
\end{align*}

\begin{align*}
X &= A + (n-1) B + P + 2(n-2) Q + \frac{1}{2}(n-2)(n-3) R
\tag{A.6}
\end{align*}

\begin{align*}
Y &= A + (n-1) B - P - 2(n-2) Q - \frac{1}{2}(n-2)(n-3) R
\tag{A.7}
\end{align*}

\begin{align*}
Z &= 2(n-1)\{2 C + (n-2) D\}^2
\tag{A.8}
\end{align*}

\begin{align*}
\lambda_1 &= \frac{1}{2}\left\{A - B + P - 4 Q + 3 R \pm \sqrt{(A-B-P+4Q-3R)^2 - 8(C-D)^2}\right\} 
\tag{A.9}
\end{align*}

\section{Eigenvalue 2}

\begin{align*}
A a + (n-1) B b + C c (n-1) + \frac{1}{2} D d (n-1)(n-2) &= \lambda_2 a 
\tag{A.10}
\end{align*}

\begin{align*}
a + (n-1) b &= 0, \quad c + \frac{1}{2}(n-2) d = 0 
\tag{A.11}
\end{align*}

\begin{align*}
\left(A - \lambda_2 - B\right) a + (n-1)(C-D) c &= 0 
\tag{A.12}
\end{align*}

\begin{align*}
a C+b C+(n-2) D b+P c+(n-2) Q c+(n-2) Q d+\frac{1}{2}(n-2)(n-3) R d=\lambda_{2} c
\tag{A.13}
\end{align*}

\begin{align*}
\frac{n-2}{n-1}(C-D) a + \left\{P + (n-4) Q - (n-3) R - \lambda_2\right\} c &= 0 
\tag{A.14}
\end{align*}

\begin{align*}
\lambda_2 &= \frac{1}{2}\left(X \pm \sqrt{Y^2 + Z}\right) 
\tag{A.15}
\end{align*}

\begin{align*}
X &= A - B + P + (n-4) Q - (n-3) R 
\tag{A.16}
\end{align*}

\begin{align*}
Y &= A - B - P - (n-4) Q + (n-3) R 
\tag{A.17}
\end{align*}

\begin{align*}
Z &= 4(n-2)(C-D)^2 
\tag{A.18}
\end{align*}

\section{Eigenvalue 3}

\begin{align*}
2 a + (n-2) b &= 0, \quad c + 2(n-2) d + \frac{1}{2}(n-2)(n-3) e = 0 
\tag{A.19}
\end{align*}

\begin{align*}
a x+a y+(n-2) b y=0, \qquad c v+(n-2) d v=0, \qquad (n-2) d w+\frac{1}{2}(n-2)(n-3) e w=0
\tag{A.20}
\end{align*}

\begin{align*}
a - b &= 0, \quad c + (n-2) d = 0, \quad d + \frac{1}{2}(n-3) e = 0 
\tag{A.21}
\end{align*}

\begin{align*}
P c + 2(n-2) Q d + \frac{1}{2}(n-2)(n-3) R e &= \lambda_3 c 
\tag{A.22}
\end{align*}

\begin{align*}
\lambda_3 &= P - 2 Q + R 
\tag{A.23}
\end{align*}

\end{document}
