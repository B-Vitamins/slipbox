\documentclass[8pt]{beamer}
\usetheme{default}
\setbeamertemplate{bibliography item}{\insertbiblabel}
\setbeamertemplate{itemize items}[square]
\setbeameroption{hide notes} % Only slides
%\setbeameroption{show only notes} % Only notes
%\setbeameroption{show notes on second screen=right} % Both
\usepackage{graphicx}
\usepackage{longtable}
\usepackage{wrapfig}
\usepackage{rotating}
\usepackage[normalem]{ulem}
\usepackage{amsmath}
\usepackage{amssymb}
\usepackage{capt-of}
\usepackage{hyperref}
\usepackage{listings}
\usepackage{centernot}
\usepackage{amsmath}
\usepackage{multimedia}
\usepackage{wrapfig}
\usepackage{biblatex}
\addbibresource{references.bib}
\hypersetup{
 pdfauthor={Ayan Das},
 pdftitle={Inherent structures of the deep Boltzmann machine},
 pdfkeywords={Deep Boltzmann machines, Inherent structures, Spin glass, Complexity, Representation capacity, Neural architecture search, Energy-based model},
 pdfsubject={Use concepts and techniques, like complexity and inherent structures, from the theory of spin glasses in statistical physics to study the representation capacity of energy-based machine learning models such as the deep Boltzmann machine},
 pdfcreator={Emacs 29.3 (Org mode 9.6.28)}, 
 pdflang={English}
}

\newcommand\mitdbar{\text{\ulcshape\slshape đ}}

\title{Inherent structures of the deep Boltzmann machine}
\date{\today}
\author{Ayan Das}
\institute{Indian Institute of Science (IISc), Bengaluru}

\begin{document}

\maketitle

%================================= Frame 0 ==================================%
\begin{frame}{Outline}
\tableofcontents
\end{frame}

%================================= Section 1 ==================================%
\section{Introduction}
\label{sec:orgf569f0a}

%================================= Frame 1 ==================================%
\begin{frame}[label={sec:org101ecb5}]{Deep Boltzmann Machines (DBMs)}
\begin{itemize}
\item \alert{Deep Boltzmann machines} (DBMs) \cite{salakhutdinov2009deep} are statistical models for parametric inference. They are foundational to deep learning and have been applied to numerous pattern recognition tasks.
\item Nevertheless, we don't have definite answers to fundamental questions like:
\end{itemize}
\begin{enumerate}
\item \emph{How does a DBM's architecture relate to the probability distributions it can represent?}
\linebreak
\item \emph{How to determine the optimal architecture for a given application?}
\linebreak
\item \emph{Why does one architecture work better than another for a specific application?}
\linebreak
\end{enumerate}
\begin{itemize}
\item Due to these gaps, the \emph{design} and \emph{tuning} of DBMs is trial and error, guided by intuitions lacking universal theoretical grounding.
\end{itemize}
\end{frame}

%================================= Frame 2 ==================================%
\begin{frame}[label={sec:org15d42bb}]{Why need depth? Why need width?}
\begin{theorem}[Le Roux and Bengio 2008]
A DBM with \(k+1\) hidden units can approximate any distribution over \(\{0,~1\}^n\) where \(k\) is the number of input vectors with non-zero probability \cite{leroux2008representational}.
\hfill \blacksquare
\label{org8221a4c}
\end{theorem}
\begin{theorem}[Montufar 2014]
A DBM with a visible layer of \(n\) units and \(L\) hidden layers of \(n\) units (lean networks) is a universal approximator of probability distributions on the visible layer if \cite{montufar2014deep}
\[
L \geq \frac{2^n-(n+1)}{n(n+1)}.
\]
\hfill \blacksquare
\label{orgd88c08b}
\end{theorem}
\begin{theorem}[Montufar 2017]
A single hidden layer DBM with less that
\[
\bigg[\frac{2(\log (v)+1)}{v+1}\bigg] 2^{v}-1
\]
hidden binary variables can approximate any distribution of \(v\) visible binary variables arbitrarily well. \cite{montufar2017hierarchical}.
\hfill \blacksquare
\label{org23a4fcc}
\end{theorem}
\end{frame}

%================================= Frame 3 ==================================%
\begin{frame}[label={sec:org8b22640}]{Why need depth? Why need width?}
\begin{enumerate}
\item If a sufficiently wide and shallow single hidden layer DBM is a universal approximator, \emph{why add layers at all}?
\linebreak
\item If a sufficiently deep and lean multiple hidden layer DBM is a universal approximator, \emph{why make layers wide at all}?
\linebreak
\end{enumerate}
\linebreak
The previous theorems are \alert{declarative results} in which:
\begin{itemize}
\item optimality concerns are secondary,
\item require parameters that are \emph{exponential} in the number of input variables,
\item the bounds can sometimes be very loose \cite{bansal2018using}.
\linebreak
\end{itemize}
For practical applications, we need \alert{imperative results} that:
\begin{itemize}
\item address how \emph{realistic} architectures relate to model capacity and representation,
\item provide rigorous prescriptions for \emph{building optimal architectures}.
\linebreak
\end{itemize}
The emerging field of \emph{Neural Architecture Search} (NAS) \cite{ren2021comprehensive} is aimed at generating results of the second kind. Here, we approach this problem from the perspective of \emph{statistical physics} with a focus on DBMs.
\end{frame}

%================================= Section 2 ==================================%
\section{Related Work}
\label{sec:org5622510}

%================================= Frame 4 ==================================%
\begin{frame}[label={sec:org622b21e}]{Connection with physics: Markov random fields}
\begin{definition}[Undirected graph]
An undirected graph is a tuple \(G=(V, E)\), where \(V\) is a finite set of nodes and \(E\) is a set of undirected edges. An edge consists out of a pair of nodes from \(V\).
\hfill \blacksquare
\label{orgf63e7b3}
\end{definition}
\linebreak
\begin{definition}[Local Markov property]
Let \(G=(V, E)\) be an undirected graph. Suppose that each node \(v \in V\) is associated a random variable \(X_{v}\) taking values in a state space \(\Lambda\) for  all \(v \in V\). For any node \(v \in V\) and its neighborhood \(\mathcal{N}_{v} = \{w : (v, ~ w) \in E\}\) such that \(r \in V \backslash \mathcal{N}_{v}\) is a not a neighbor of \(v\), the variables \((X_{v})_{v \in V}\) and \((X_{r})_{r \in V\backslash \mathcal{N}_{v}}\) are conditionally independent given \((X_{n})_{n \in \mathcal{N}_{v}}\) \cite{fischer2012introduction}, i.e.,
\[
\forall~\boldsymbol{x} \in \Lambda^{\lvert V \rvert} \quad p \big((x_{v})_{v \in V} \mid (x_{t})_{t \in \mathcal{N}_{v} \cup V \backslash \mathcal{N}_{v}} \big) = p \big((x_{v})_{v \in V} \mid\left(x_{t}\right)_{t \in \mathcal{N}_{v}}\big).
\]
\hfill \blacksquare
\label{orgc0963b4}
\end{definition}
\linebreak
\begin{definition}[Markov random field]
The random variables \(\boldsymbol{X}=\left(X_{v}\right)_{v \in V}\) are said to form a Markov random field (MRF) if the joint probability distribution \(p (\boldsymbol{X})\) fulfills the (local) Markov property with respect to the graph \(G\). \cite{fischer2012introduction}
\hfill \blacksquare
\label{org2322992}
\end{definition}
\end{frame}

%================================= Frame 5 ==================================%
\begin{frame}[label={sec:org440a44b}]{Connection with physics: The Hammersley-Clifford theorem}
The joint distribution \(p\) of \emph{any} arbitrary MRF is the \emph{Gibbs-Boltzmann distribution} for an appropriately chosen \emph{energy function} \(E(\mathbf{x})\) given \(p(\mathbf{x}) > 0 \quad \forall~\mathbf{x}\). This fundamental result is the basis of all \alert{energy-based models}.
\begin{definition}[Clique]
Let \(G=(V, E)\) be an undirected graph. A clique \(C\) is a subset of \(V\) such that if \(w \in C\) and \(v \in C\) then \((w,~v) \in E\). All nodes in a clique are pairwise connected.
\hfill \blacksquare
\label{orge51d0be}
\end{definition}
\begin{theorem}[Hammersley and Clifford]
Let \(C\) denote the (maximal) cliques of an undirected graph \(G\). A strictly positive distribution \(p\) satisfies the conditional independence properties (or local Markov property) of \(G\) if and only if \(p\) admits the following factorization  \cite{clifford1990markov}:
\[
p(\mathbf{x}) = \mathcal{Z}^{-1} \prod_{C} \psi_{C} (\mathbf{x}_C) = \mathcal{Z}^{-1} \exp \big[- E(\mathbf{x})\big]
\]
where
\[
E(\mathbf{x}) = - \sum_{C} \ln \psi_{C} (\mathbf{x}_{C}), \quad \mathcal{Z} = \sum_{\mathbf{x}} \exp \big[- E(\mathbf{x})\big].
\]
The variables \(\mathbf{x}_{C}\) are members of the clique \(C\) and \(\psi_{C}\) is an arbitrary function of \(\mathbf{x}_{C}\).
\hfill \blacksquare
\label{org91b2496}
\end{theorem}
\end{frame}

%================================= Frame 6 ==================================%
\begin{frame}[label={sec:org92ea9e6}]{Connection with physics: Gibbs-Boltzmann distribution}
The DBM is an \emph{energy-based model} closely linked with the \emph{Ising model} with random interactions. In fact, the DBM is a special case of the general \emph{Boltzmann machine} which is completely equivalent to the \emph{Sherrington-Kirkpatrick model} \cite{sherrington1975solvable}, another widely studied model in statistical physics.
\(\boldsymbol{\sigma}\) is a collection of \emph{Ising spins} that denotes the \emph{state} of a \(L\) hidden layer DBM:
  \[
  \boldsymbol{\sigma} \equiv (\sigma_{il} \in \{\pm 1\})_{l=0 \ldots L}^{i = 1 \ldots N_{l}} \in \mathcal{S}, \quad \mathcal{S} \equiv \{\pm 1\}^{N_{0}} \times \{\pm 1\}^{N_{1}} \times \cdots \times \{\pm 1\}^{N_{L}}, \quad |\mathcal{S}| = \prod_{l=0}^{L} 2^{N_{l}}.
  \]
The energy function or \emph{Hamiltonian} \(H_{\boldsymbol{J}} (\boldsymbol{\sigma})\) of this DBM with \(L\) hidden layers and \emph{couplings} \(\boldsymbol{J}\) has pairwise interactions:
  \[
  H_{\boldsymbol{J}} (\boldsymbol{\sigma}) \equiv - \sum_{l=0}^{L-1} \sum_{i=1}^{N_l} \sum_{j=1}^{N_{l+1}} \sigma_{il} J_{ijl} \sigma_{j(l+1)}, \quad \boldsymbol{J} \equiv (\boldsymbol{J_{l}})_{l=0 \ldots L-1}, \quad \boldsymbol{J}_{l} \equiv (J_{ijl})_{i=0 \ldots N_{l}}^{j=1 \ldots N_{l+1}}.
  \]
The \emph{Gibbs-Boltzmann distribution} assigns probabilities to \(\boldsymbol{\sigma} \in \mathcal{S}\):
  \[
  p (\boldsymbol{\sigma}) \equiv \mathcal{Z}^{-1} \exp \big[- H_{\boldsymbol{J}} (\boldsymbol{\sigma}) \big] \quad \mathcal{Z} \equiv \sum_{\boldsymbol{\sigma}} \exp \big[- H_{\boldsymbol{J}} (\boldsymbol{\sigma}) \big].
  \]
\end{frame}

%================================= Frame 7 ==================================%
\begin{frame}[label={sec:org0280aaf}]{Inherent structures}
Randomness in \emph{couplings} often introduce \emph{frustration} in \emph{microscopic interactions}. Interaction of a spin \(\sigma\) and its neighbors \(\mathcal{N}(\sigma)\) is frustrated if
\begin{equation*}
\min_{\{\sigma, ~ \mathcal{N}(\sigma)\}} \bigg(- \sum_{\tau \in \mathcal{N} (\sigma)} \sigma J_{\sigma \tau} \tau \bigg) \neq \sum_{\tau \in \mathcal{N} (\sigma)} \min_{\{\sigma, \tau\}} (- \sigma J_{\sigma \tau} \tau).
\end{equation*}
Frustration can sometimes lead to \emph{metastability} where the energy function has a large number of \emph{local minima} or inherent structures.
\begin{definition}[Inherent structures, Stillinger and Weber]
\item The \alert{inherent structures} of a configuration space \(\mathcal{S}\) of Ising spins \(\boldsymbol{\sigma}\) is the set of configurations \(\{ \boldsymbol{\sigma}_{\text{IS}} \}\) for which the energy function \(H_{\boldsymbol{J}}\) is at a local minimum, i.e., \(\delta H_{\boldsymbol{J}} (\boldsymbol{\sigma}_{\mathrm{IS}}) \geq 0\) \cite{Stillinger1983InherentSI}.
\hfill \blacksquare
\end{definition}
\begin{definition}[Basin of attraction]
\item The \alert{basin of attraction} \(\mathcal{B}(\boldsymbol{\sigma}_{\text{IS}})\) of an inherent structure \(\boldsymbol{\sigma}_{\text{IS}}\) is defined as
\[
\mathcal{B} (\boldsymbol{\sigma}_{\text{IS}}) \equiv \{ \boldsymbol{\sigma}\,:\, G(\boldsymbol{\sigma}) = \boldsymbol{\sigma}_{\text{IS}}  \}
\]
where \(G\) denotes a \emph{steepest descent minimization} \cite{schnabel2017dynamic}.
\hfill \blacksquare
\end{definition}

\end{frame}

%================================= Frame 8 ==================================%
\begin{frame}[label={sec:orgf480467}]{Inherent structures capacity (ISC)}
A configuration space \(\mathcal{S}\) \emph{decomposes} into a set of basins \(\mathcal{S} \equiv \{\mathcal{B}(\boldsymbol{\sigma}_{\mathrm{IS}})\}_{\boldsymbol{\mathrm{IS}}}\). The inherent structure \(\boldsymbol{\sigma}_{\text{IS}}\) is the \emph{local attractor} of dynamics within its basin \(\mathcal{B}(\boldsymbol{\sigma}_{\mathrm{IS}})\).
\linebreak \linebreak
\emph{Bansal, Anand, and Bhattacharyya} \cite{bansal2018using} apply the \(\mathrm{IS}\) decomposition to the configuration space of one and two hidden layer DBMs. They
\linebreak
\begin{enumerate}
\item define a derived metric, the \alert{inherent structure capacity} (\(\mathrm{ISC}\)),
\linebreak
\item use \(\mathrm{ISC}\) to design optimal DBM architectures under a budget,
\linebreak
\item show that the optimal architecture yields an \emph{order-of-magnitude} savings in parameters,
\linebreak
\item prove that in certain regimes DBMs with \(L = 2\) have superior model capacity over DBMs with \(L = 1\).
\end{enumerate}
\end{frame}

%================================= Frame 9 ==================================%
\begin{frame}[label={sec:orge513e7f}]{Definition of ISC}
Later we will present results that use conceptually different arguments but recover several of the results from \cite{bansal2018using}. Therefore, it is useful to repeat these results.
\begin{definition}[One-flip-stable state]
For an energy function \(E\) a configuration, \(\mathbf{s}^*\) is called a local minimum, also called one-flip-stable state, if \(\forall \mathbf{s} \in\left\{s: d_H\left(\mathbf{s}, \mathbf{s}^*\right)=\right.\) \(1\}, E(\mathbf{s})-E\left(\mathbf{s}^*\right)>0\) (equivalently \(P(\mathbf{s})<P\left(\mathbf{s}^*\right)\)).
\hfill \blacksquare
\label{org3398b6f}
\end{definition}
\begin{definition}[Inherent structure capacity]
For an \(L\)-layered DBM with \(m_1, \ldots, m_L\) hidden units and \(n\) visible units we define the Inherent Structure Capacity (ISC), denoted by \(C(n, m_1, \ldots, m_L)\), to be the logarithm (divided by \(n\)) of the expected number of modes of all possible distributions generated over the visible units by the DBM.

\[
C(n, m_1, \ldots, m_L) = \frac{1}{n} \log_2 \mathbb{E}_{\theta} 
\left[ \lvert \{ v : \mathcal{H}(v) \geq 1 \} \rvert \right] \tag{8}
\]

where \(\mathcal{H}(v) \triangleq \left\{ \{h_l\}_{l=1}^L | (v, \{h_l\}_{l=1}^L) \text{ is one-flip stable state} \right\}\).
\hfill \blacksquare
\label{org8361ad5}
\end{definition}

\end{frame}

%================================= Frame 10 ==================================%
\begin{frame}[label={sec:orgc351d56}]{ISC for RBM and DBM}
\begin{corollary}[L=1, wide hidden layer]
For the set \(\mathrm{RBM}_{n,m}\) (\(n > 0\), \(m > 0\)), we have for the \mathrm{ISC} \(C(n, m)\)
\[
\lim_{m \to \infty} C(n, m) = \log_2 1.5 = 0.585.
\]
\hfill \blacksquare
\label{org49aac17}
\end{corollary}
\begin{corollary}[L=2, layer 1 wide, layer 2 narrow]
For a \(\mathrm{DBM}_{n, m_1, m_2}\) (\(n > 0\), \(m_1 > 0\), \(m_2 \geq 0\)), if \(\alpha_1=m_1/n > 1/\beta\) and \(\alpha_2 = m_2/n <\beta\), where \(\beta = 0.05\), then
\[
\mathcal{C}\left(n,~m_1,~m_2\right) \leq\left(1+\alpha_2\right) \log _2(1.5). 
\]
\hfill \blacksquare
\label{org523335a}
\end{corollary}
\begin{corollary}[Maximum ISC under a budget]
For a \(\mathrm{DBM}_{n, m_1, m_2}\) (\(n>0\), \(m_1>0\), \(m_2\geq0\)), if there is a budget of \(cn^2\) on the total number of parameters then \(\max _{\alpha_1, \alpha_2} \mathcal{C}\left(n, \alpha_1, \alpha_2\right) \leq \tilde{U}\left(n, \alpha_1^*, \alpha_2^*\right)\) where
\begin{align*}
\tilde{U}\left(n,~\alpha_1^*,~\alpha_2^*\right)= \begin{cases}\min \left(1, \sqrt{c} \log _2(1.29)\right) & \text { if } c \geq 1 \\ c \log_2 \big[1-(1/2) \operatorname{erf}\big(-\sqrt{1/ (\pi c)}\big) \big] & \text { if } c<1\end{cases}
\end{align*}
\hfill \blacksquare
\label{orga2415bf}
\end{corollary}
\end{frame}

%================================= Frame 11 ==================================%
\begin{frame}[label={sec:org2313505}]{Network design under a budget}
\begin{tabular}{|c|c|c|}
\hline
\textbf{Regime} & \textbf{ISC} & \textbf{Recommendation} \\
\hline \hline
\(\alpha_1 > \frac{1}{\gamma}, \alpha_2 < \gamma\) & \((1 + \alpha_2) \log_2(1.5)\) & ISC determined only by \(\alpha_2\). \\
 & & Multilayering recommended. \\
\hline
& & Budget on parameters \(p < cn^2\). \\
\(\alpha_1(1 + \alpha_2) = c\) & \(\min\bigg(1, \frac{c}{\sqrt{\log_2(1.29)}}\bigg)\) & Multilayering recommended. \\
(c \geq 1) & & The optimal choice is \(\alpha_1 = \sqrt{c}\). \\
\hline
& & Budget on parameters \(p < n^2\), \\
\(\alpha_1(1 + \alpha_2) = c\) & \(c \log_2 \frac{1}{2} \left[1 + \operatorname{erf}\left(-\frac{1}{\sqrt{\pi c}}\right)\right]\) & Multilayering \textbf{not} recommended. \\
\((c < 1)\) & & The optimal choice is \(\alpha_1 = c\). \\
\hline
\end{tabular}
\linebreak \linebreak
Reproduced from \cite{bansal2018using} with slight modifications. Recall that for \(\mathrm{DBM}_{n,m_{1}, m_{2}}\), \(\alpha_{1} = m_{1} / n\), and \(\alpha_{2} = m_{2} / n\). \(\gamma = 0.05\) is obtained by simulating the asymptotics of \(1 - (1/2) \operatorname{erf} \big(- \sqrt{x/ \pi} \big)\). 
\end{frame}

%================================= Frame 12 ==================================%
\begin{frame}[label={sec:org7702ed6}]{Limitations}
The analysis done by \emph{Bansal et. al.} has some limitations:
\linebreak
\begin{enumerate}
\item the analysis is restricted to DBMs with 1 (\(L = 1\)) and 2 (\(L=2\)) hidden layers,
\linebreak
\item for DBMs with \(L = 2\), the analysis is restricted to a regime where layer 1 is wide, and layer 2 is narrow,
\linebreak
\item there is no way to determine the numerical value of the \(\mathrm{ISC}\) for arbitrary architectures.
\linebreak
\end{enumerate}
Our work aims to overcome these limitations. We will use a different approach using techniques detailed in \cite{singh1995fixed}, \cite{gutfreund1988attractors}, \cite{tanaka1980analytic}.
\end{frame}

%================================= Section 3 ==================================%
\section{Problem statement}
\label{sec:org76f4fa2}

%================================= Frame 13 ==================================%
\begin{frame}[label={sec:org621e956}]{Problem statement}
\begin{itemize}
\item The energy function of a DBM with \(L\) hidden layers:
  \[
  H_{\boldsymbol{J}} (\boldsymbol{\sigma}) = - \sum_{l=0}^{L-1} \sum_{i=1}^{N_l} \sum_{j=1}^{N_{l+1}} \sigma_{il} J_{ijl} \sigma_{j(l+1)}.
  \]
\linebreak
\item Let \(\boldsymbol{N} \equiv (N_{l})_{l=0\ldots L}\) be a vector whose components denote the number of units in each layer. Call this vector the DBM's \alert{architecture}.
\linebreak
\item Assume that the couplings \(\boldsymbol{J}\) are sampled from a Gaussian distribution \cite{nishimori2001spsg}:
\[
  p_{J}(J_{ijl}) \equiv \big(\widehat{N}_{l} / 2 \pi J^2 \big)^{1/2} \exp \big[- (\widehat{N}_{l} / 2 J^{2}) \thinspace J_{ijl}^{2} \thinspace \big] \qquad \widehat{N} \equiv \sqrt{N_{l} N_{l+1}}.
  \]
where \(J\) is a positive real number and \((\widehat{N}_{l})_{l=0 \ldots L-1}\) are the pairwise geometric means of the number of units in adjacent layers \cite{hartnett2018replica}.
\linebreak
\item Let \(\mathcal{N}_{s}\) denote the \emph{number} of \(\mathrm{IS}\) in configuration space \(\mathcal{S}\) of a DBM. Consider an \emph{ensemble} of DBMs with an arbitrary but fixed architecture \(\boldsymbol{N}\). Find an expression for the \emph{inherent structure capacity} \(\mathcal{C}_{J}\) for this ensemble defined as
\begin{equation*}
\mathcal{C}_{J} (\boldsymbol{N}) \equiv N_{0}^{-1} \ln \langle \mathcal{N}_{s} \rangle_{J}.
\end{equation*}
\end{itemize}
\end{frame}

%================================= Section 4 ==================================%
\section{Results}
\label{sec:orgf02a33e}

%================================= Frame 14 ==================================%
\begin{frame}[label={sec:org53dbd1a}]{Preliminaries: geometric parameters}
\begin{definition}[Total spin number]
The \emph{total spins number} \(N\) is the sum of the number of spins across all layers \(N \equiv \sum_{l} N_l\). \hfill \blacksquare
\label{org1e129a9}
\end{definition}
\begin{definition}[Proportion]
The \emph{proportion} \((\alpha_{l})_{l=0\ldots L}\) of the units in layer \(l\) relative to the visible layer \((l = 0)\) is defined as \(\alpha_l \equiv N_l /N_{0}\). \hfill \blacksquare
\label{org115ac25}
\end{definition}
\begin{definition}[Inter-layer ratio]
The \emph{inter-layer ratios} \((\gamma_{l})_{l=0 \ldots L-1}\) and \((\nu_{l})_{l=1 \ldots L}\) are defined as
\begin{align*}
\gamma_{l}^{2} \equiv N_{l+1} / N_{l}, \quad \nu_{l}^{2} \equiv N_{l-1} / N_{l}.
\end{align*}
\hfill \blacksquare
\label{orga0dc257}
\end{definition}
The definitions above yield the following identities
\begin{align*}
\alpha_{0} \equiv 1, \qquad \gamma_{l}^{-1} \equiv \nu_{l+1}, \quad \nu_{l}^{-1} \equiv \gamma_{l-1}, \qquad \widehat{N}_{l} = N_{0} \sqrt{\alpha_l \alpha_{l+1}}.
\end{align*}
\end{frame}

%================================= Frame 15 ==================================%
\begin{frame}[label={sec:org3d35695}]{Preliminaries: geometric parameters}
\begin{figure}[h]
  \centering
  \includegraphics[width=.7\linewidth]{/home/b/.local/images/dbm.pdf}
  \caption{An illustration of a DBM with \(L = 3\) and \(N = 10\). The two colors are representations for the values \(\sigma_{il} = \pm 1\). The geometric parameters are illustrated.}
\end{figure}
\end{frame}

%================================= Frame 16 ==================================%
\begin{frame}[label={sec:org0361a7a}]{Preliminaries: single-site energies}
We will work with a definition of \(\mathrm{IS}\) in terms of the \alert{single-site energies}, given by \emph{Tanaka and Edwards} \cite{tanaka1980analytic}. It is equivalent to how \emph{Stillinger and Weber} define \(\mathrm{IS}\).
\linebreak
\begin{definition}[Single-site energy]
The \emph{single-site energy} of spin \(i\) in layer \(l\), \(\sigma_{il}\), is defined as
\[
\lambda_{i l} \equiv \sigma_{i l} \bigg( \sum_{j=1}^{N_{l+1}} J_{ijl} \sigma_{j (l+1)} (1 - \delta_{lL}) + \sum_{j=1}^{N_{l-1}} J_{ji(l-1)} \sigma_{j (l-1)} (1 - \delta_{l0}) \bigg) \tag{5}
\]
Every spin has an associated \emph{single-site energy} so we have \((\lambda_{il})_{l=0 \ldots L}^{i= 1 \ldots N_{l}}\).
\hfill \blacksquare
\label{orga258f92}
\end{definition}
\linebreak
\begin{definition}[Inherent structure, Tanaka and Edwards]
The inherent structures of the DBM are the configurations \((\boldsymbol{\sigma})_{\text{IS}}\) for which all the single site energies are strictly positive, i.e., \(\lambda_{il} > 0\) for all values of \(i\) and \(l\). In other words, \((\boldsymbol{\sigma})_{\text {IS}}\) are \emph{stable against the flips of a single spin}.
\hfill \blacksquare
\label{org10c8d87}
\end{definition}
\end{frame}

%================================= Frame 17 ==================================%
\begin{frame}[label={sec:org3af19bb}]{Main result}
\begin{theorem}[ISC for a DBM with L hidden layers]
Consider an ensemble of DBMs with a fixed architecture \(\boldsymbol{N}\) and couplings drawn from a Gaussian prior \(p_{J}(J_{ijl})\) as previously defined. The inherent structure capacity \(\mathcal{C}_{J} (\boldsymbol{N}) \equiv N_{0}^{-1} \ln \langle \mathcal{N}_{s} \rangle_{J}\) for this ensemble is given by

\begin{align*}
&\mathcal{C}_{J} (\boldsymbol{N}) = \underset{\{(x_{l}, y_{l})_{l}\}}{\operatorname{saddle}} \thinspace \frac{1}{2} \bigg \{- \sum_{l=0}^{L-1} \sqrt{\alpha_{l} \alpha_{l+1}} \thinspace \big(x_{l}^{2} + y_{l}^{2} \big) + \alpha_{L} \ln \bigg [1 + \operatorname{erf} \bigg(\frac{- x_{L-1} - i y_{L-1}}{\sqrt{2 \nu_{L}}} \bigg) \bigg ] \\
&+ \sum_{l=1}^{L-1} \alpha_{l} \ln \bigg [1 + \operatorname{erf} \bigg(\frac{- (x_{l} + x_{l-1}) + i (y_{l} - y_{l-1})}{\sqrt{2 (\gamma_{l} + \nu_{l})}} \bigg) \bigg] + \ln \bigg [1 + \operatorname{erf} \bigg(\frac{- x_{0} + i y_{0}}{\sqrt{2 \gamma_{0}}} \bigg) \bigg] \bigg \}.
\end{align*}
\hfill \blacksquare
\label{orgc8a0373}
\end{theorem}
This formula improves upon the ISC derived by \cite{bansal2018using} in the following ways:
\begin{enumerate}
\item applies to DBMs with an arbitrary number of hidden layers \(L\) (as opposed to \(L = 2\)),
\item applies to DBMs with arbitrary proportions \((\alpha_{i})_{i=1\ldots L}\) (as opposed to \(\alpha_{1} \gg 1\) and \(\alpha_{2} \ll 1\)),
\item gives numerical values of \(\mathrm{ISC}\) for any arbitrary architecture \(\boldsymbol{N}\) (as opposed to only the optimal architecture).
\end{enumerate}
\end{frame}

%================================= Frame 18 ==================================%
\begin{frame}[label={sec:orgcd7776d}]{Step 1: Area formula}
The derivation begins with an \alert{area formula}, a special case of \alert{Kac-Rice formula} \cite{berzin2022kac}:

\begin{align*}
\mathcal{N}_{s} &= \frac{1}{2} \overbrace{\int_0^{\infty} \prod_{l=0}^L \prod_{i=1}^{N_l} \mathrm{~d} \lambda_{il}}^{\text{integral over site energies}} \overbrace{\sum_{\boldsymbol{\sigma}} \prod_{l=0}^L \prod_{i=1}^{N_l}}^{\text{spin configurations}} \\
&\qquad \times \underbrace{\delta \bigg(\lambda_{il} - \sigma_{i l} \bigg[ \sum_{j=1}^{N_{l+1}} J_{ijl} \sigma_{j (l+1)} (1 - \delta_{lL}) + \sum_{j=1}^{N_{l-1}} J_{ji(l-1)} \sigma_{j (l-1)} (1 - \delta_{l0}) \bigg] \bigg)}_{\text{spins with energy} \qquad \( \lambda_{il} \)}
\end{align*}

Using the integral representation of the \(\delta\) function, \(\delta(x)=\frac{1}{2 \pi} \int_{-\infty}^{\infty} \mathrm{d} k \exp (- \mathrm{i} k x)\), with real valued variables \((k_{il})_{l=0 \ldots L}^{i=1 \ldots N_{l}}\), we calculate the expectation value \(\langle \mathcal{N}_s \rangle_{J}\)
\begin{align*}
2 \thinspace \langle \mathcal{N}_{s} \rangle_{J} = (\mathrm{i} \pi)^{-N} &\int_0^{\infty} \prod_{ i l} \mathrm{d} \lambda_{il} \int_{-\mathrm{i}\infty}^{\mathrm{i} \infty} \prod_{i l} \mathrm{d} k_{il} \exp \bigg(\sum_{il} k_{il} \lambda_{il} \bigg) \\
&\times \exp \bigg\{\frac{1}{2} \sum_{l=0}^{L - 1} \sum_{i=1}^{N_{l}} \gamma_{l} k_{il}^{2} + \frac{1}{2} \sum_{l=0}^{L - 1} \sum_{j=1}^{N_{l+1}} \nu_{l+1} k_{j (l+1)}^{2} + \sum_{l=0}^{L-1} \frac{1}{\widehat{N}_{l}} \sum_{ij} k_{il} k_{j(l+1)} \bigg \}.
\end{align*}
\end{frame}

%================================= Frame 19 ==================================%
\begin{frame}[label={sec:orgec61b4c}]{Step 2: Hubbard-Stratonovich transformation}
Using a \alert{Hubbard-Stratonovich integral transform} \cite{hubbard1959calculation}
\[
\exp \bigg(\frac{b c}{a} \bigg) &= \frac{a}{\pi} \iint_{-\infty}^{\infty} \mathrm{d}x \mathrm{~d} y ~ \exp \big[  - a (x^{2} + y^{2}) + b (x - \mathrm{i} y) + c(x + \mathrm{i} y) \big] \quad a > 0
\]
on the factor \(\exp \big(\widehat{N}_{l}^{-1} \sum_{ij} k_{il} k_{j(l+1)} \big)\) with auxiliary variables \((x_{l})_{l=0 \ldots L-1}\) and \((y_{l})_{l=0 \ldots L-1}\) we obtain
\begin{align*}
&2 \pi \mathrm{i} \thinspace \langle \mathcal{N}_{s} \rangle_{J} = \iint_{-\infty}^{\infty} \prod_{l=0}^{L-1} \big(\widehat{N}_{l} / \pi \big)  \big( \mathrm{d} y_{l} \mathrm{d} x_{l} \big) \exp \bigg\{ - \sum_{l=0}^{L-1} \widehat{N}_{l} \big( x_{l}^{2} + y_{l}^{2} \big) \bigg\} \int_0^{\infty} \prod_{ i l} \mathrm{d} \lambda_{il} \\
&\int_{-\infty}^{\infty} \prod_{l=1}^{L-1} \prod_{i=1}^{N_{l}} \mathrm{d} k_{il} \exp \bigg\{\sum_{l=1}^{L - 1} \sum_{i=1}^{N_{l}} \bigg[- \frac{1}{2} \big( \gamma_{l} + \nu_{l} \big) k_{il}^{2} + \mathrm{i} \big[(x_{l} + x_{l-1}) - \mathrm{i} (y_{l} - y_{l-1}) + \lambda_{il} \big] k_{il} \bigg] \bigg\} \\
&\quad \int_{-\infty}^{\infty} \prod_{i=1}^{N_{0}} \mathrm{d} k_{i0} \exp \bigg\{ \sum_{i=1}^{N_{0}} \bigg(- \frac{1}{2} \gamma_{0} k_{i0}^{2}  + \mathrm{i} \big[x_{0} - \mathrm{i} y_{0} + \lambda_{i0} \big]  k_{i0} \bigg) \bigg \} \\
&\qquad \int_{-\infty}^{\infty} \prod_{i=1}^{N_{L}} \mathrm{d} k_{iL} \exp \bigg\{\sum_{i=1}^{N_{L}} \bigg( - \frac{1}{2} \nu_{L} k_{iL}^{2} + \mathrm{i} \big[ x_{L-1} + \mathrm{i} y_{L-1} + \lambda_{iL} \big] k_{iL} \bigg) \bigg \}.
\end{align*}
\end{frame}

%================================= Frame 20 ==================================%
\begin{frame}[label={sec:org6ca474e}]{Step 3: Gaussian integral}
The integral over \((k_{il})_{l=0 \ldots L}^{i=1 \ldots N_{l}}\) are now all Gaussian. After evaluation
\begin{align*}
&2 \pi \thinspace \langle \mathcal{N}_{s} \rangle_{J} = \pi^{N/2} \bigg(\frac{2}{\gamma_{0}}\bigg)^{N_{0}/2} \bigg(\frac{2}{\nu_{L}}\bigg)^{N_{L}/2} \prod_{l=1}^{L-1} \bigg(\frac{2}{\gamma_{l} + \nu_{l}} \bigg)^{N_{l}/2} \\
&\iint_{-\infty}^{\infty} \prod_{l=0}^{L-1} \big(\widehat{N}_{l} / \pi \big)  \big( \mathrm{d} y_{l} \mathrm{d} x_{l} \big)  \exp \bigg\{ - \sum_{l=0}^{L-1} \widehat{N}_{l} \big( x_{l}^{2} + y_{l}^{2} \big) \bigg\} \\
&\int_0^{\infty} \prod_{i=1}^{N_0} \mathrm{d} \lambda_{i0} \exp \bigg \{ \sum_{i=1}^{N_{0}} \bigg( -\frac{\lambda_{i0}^{2}}{2 \gamma_{0}} + \frac{\lambda_{i0} \big(- x_{0} + \mathrm{i} y_{0}\big)}{\gamma_{0}} \bigg) \bigg \} \\
&\int_0^{\infty} \prod_{i=1}^{N_L} \mathrm{d} \lambda_{iL} \exp \bigg \{  \sum_{i=1}^{N_{L}} \bigg(- \frac{\lambda_{iL}^{2}}{2 \nu_{L}} + \frac{\lambda_{iL} \big(- x_{L-1} - \mathrm{i} y_{L-1}\big)}{\nu_{L}} \bigg) \bigg\} \\
&\int_0^{\infty} \prod_{l=1}^{L-1} \prod_{i=1}^{N_l} \mathrm{d} \lambda_{il} \exp \bigg\{ \sum_{l=1}^{L - 1} \sum_{i=1}^{N_{l}} \bigg( - \frac{\lambda_{il}^{2}}{2 (\gamma_{l} + \nu_{l})} + \frac{\lambda_{il} \big[- (x_{l} + x_{l-1}) + \mathrm{i} (y_{l} - y_{l-1})\big]}{(\gamma_{l} + \nu_{l})} \bigg) \bigg \} \\
&\exp \bigg\{ - \sum_{l=1}^{L - 1} \sum_{i=1}^{N_{l}} \frac{\big[(x_{l} + x_{l-1}) - \mathrm{i} (y_{l} - y_{l-1})\big]^{2}}{2 (\gamma_{l} + \nu_{l})} - \sum_{i=1}^{N_{0}} \frac{\big(x_{0} - \mathrm{i} y_{0}\big)^{2}}{2 \gamma_{0}} - \sum_{i=1}^{N_{L}} \frac{\big(x_{L-1} + \mathrm{i} y_{L-1}\big)^{2}}{2 \nu_{L}} \bigg\}.
\end{align*}
\end{frame}

%================================= Frame 21 ==================================%
\begin{frame}[label={sec:org60631d9}]{Step 4: Half-integral}
The integral over the single-site energies \((\lambda_{il})_{l=0 \ldots L}^{i=1 \ldots N_{l}}\) are now all Gaussian but the interval is \([0,~\infty)\). Using the result
\[
\int_0^{\infty} \exp \bigg(-\frac{1}{2} a x^2+b x\bigg) d x=\bigg(\frac{\pi}{2 a}\bigg)^{\frac{1}{2}} \exp \bigg(\frac{b^2}{2 a}\bigg)\bigg[1+\operatorname{erf}\bigg(\frac{b}{\sqrt{2 a}}\bigg)\bigg] \qquad a > 0.
\]
we evaluate the half integrals to obtain
\begin{align*}
&2 \thinspace \langle \mathcal{N}_{s} \rangle_{J} = \bigg(\prod_{l=0}^{L-1} \frac{N_{0} \alpha_{l} \alpha_{l+1}}{\pi} \bigg) \int_{-\infty}^{\infty} \mathrm{d} y_{l} \int_{-\infty}^{\infty} \mathrm{d} x_{l} \exp \bigg\{- N_{0} \sum_{l=0}^{L-1} \sqrt{\alpha_{l} \alpha_{l+1}} \big( x_{l}^{2} + y_{l}^{2} \big)\bigg\} \\
&\quad \times \exp \bigg \{ N_{0} \ln \bigg [1 + \operatorname{erf} \bigg(\frac{- x_{0} + \mathrm{i} y_{0}}{\sqrt{2 \gamma_{0}}} \bigg) \bigg ] +  N_{0} \thinspace \alpha_{L} \ln \bigg [1 + \operatorname{erf} \bigg(\frac{- x_{L-1} - \mathrm{i} y_{L-1}}{\sqrt{2 \nu_{L}}} \bigg) \bigg ]  \bigg\} \\
&\quad \quad \times \exp \bigg \{ N_{0} \sum_{l=1}^{L-1} \alpha_{l} \ln \bigg [1 + \operatorname{erf} \bigg(\frac{- (x_{l} + x_{l-1}) + \mathrm{i} (y_{l} - y_{l-1})}{\sqrt{2 (\gamma_{l} + \nu_{l})}} \bigg) \bigg ]  \bigg\}.
\end{align*}
\end{frame}

%================================= Frame 22 ==================================%
\begin{frame}[label={sec:org24556e3}]{Step 5: Steepest descent approximation}
The integral can now be evaluated using the method of \alert{steepest descent} \cite{kardar2007spop} in the limit \(N_{0} \to \infty\)
\[
\lim_{N_{0} \to \infty} N_{0}^{-1} \ln \left \langle \mathcal{N}_{s} \right \rangle_{J} = & \lim_{N_{0} \to \infty} \bigg[ \mathcal{C}_{J} (\boldsymbol{N}) - \frac{1}{2N_{0}} \ln \bigg(\frac{N_{0} \lvert \mathcal{C}_{J}^{\prime \prime} (\boldsymbol{N}) \rvert}{2 \pi} \bigg) + \mathcal{O} \bigg(\frac{1}{N_{0}^{2}} \bigg) \bigg],
\]
where we have identified the right hand side as the leading order behavior of the \(\mathrm{ISC}\) in the limit \(N_{0} \to \infty\)
\begin{align*}
&\mathcal{C}_{J} (\boldsymbol{N}) = \underset{\{(x_{l}, y_{l})_{l}\}}{\operatorname{saddle}} \thinspace \frac{1}{2} \bigg \{- \sum_{l=0}^{L-1} \sqrt{\alpha_{l} \alpha_{l+1}} \thinspace \big(x_{l}^{2} + y_{l}^{2} \big) + \alpha_{L} \ln \bigg [1 + \operatorname{erf} \bigg(\frac{- x_{L-1} - \mathrm{i} y_{L-1}}{\sqrt{2 \nu_{L}}} \bigg) \bigg ] \\
&+ \sum_{l=1}^{L-1} \alpha_{l} \ln \bigg [1 + \operatorname{erf} \bigg(\frac{- (x_{l} + x_{l-1}) + \mathrm{i} (y_{l} - y_{l-1})}{\sqrt{2 (\gamma_{l} + \nu_{l})}} \bigg) \bigg] + \ln \bigg [1 + \operatorname{erf} \bigg(\frac{- x_{0} + \mathrm{i} y_{0}}{\sqrt{2 \gamma_{0}}} \bigg) \bigg] \bigg \}.
\end{align*}
\end{frame}

%================================= Frame 23 ==================================%
\begin{frame}[label={sec:orgc279062}]{Step 6: Fixed point equations}
With
\begin{align*}
&\theta_{l} \equiv
\begin{cases}
\frac{1}{\sqrt{2 \gamma_0}}, & l = 0 \\
\frac{1}{\sqrt{2 (\gamma_{l} + \nu_{l})}}, & 0 < l < L - 1 \\
\frac{1}{\sqrt{2 \nu_{L}}}, & l = L - 1\\
\end{cases}
&
\omega_{l} &\equiv
\begin{cases}
- x_{0} + \mathrm{i} y_{0}, & l = 0 \\
- (x_{l} + x_{l-1}) + \mathrm{i} (y_{l} - y_{l-1}), & 0 < l < L - 1\\
- x_{L-1} - \mathrm{i} y_{L-1}, & l = L - 1\\
\end{cases}
\end{align*}
we need to iterate the following equations to obtain the saddle \(\big\{(x_{l}^{\text{max}} , y_{l}^{\text{max}})_{l=0 \ldots L-1} \big \}\)
\begin{align*}
x_{l} &= - \frac{\gamma_{l}^{-1} \theta_{l}}{\sqrt{2 \pi}} \exp \big[- (\theta_{l} \thinspace \omega_{l})^{2} \big] \bigg[1 + \operatorname{erf}\big( \theta_{l} \thinspace \omega_{l} \big) \bigg]^{-1} \\
&\quad - \frac{\gamma_{l} \theta_{l+1}}{\sqrt{2 \pi}} \exp \big[- (\theta_{l+1} \thinspace \omega_{l+1})^{2} \big] \bigg[1 + \operatorname{erf}\big( \theta_{l+1} \thinspace \omega_{l+1} \big) \bigg]^{-1},
\end{align*}
\begin{align*}
\mathrm{i} y_{l} &= - \frac{\gamma_{l}^{-1} \theta_{l}}{\sqrt{2 \pi}} \exp \big[- (\theta_{l} \thinspace \omega_{l})^{2} \big] \bigg[1 + \operatorname{erf}\big( \theta_{l} \thinspace \omega_{l} \big) \bigg]^{-1} \\
&\quad + \frac{\gamma_{l} \theta_{l+1}}{\sqrt{2 \pi}} \exp \big[- (\theta_{l+1} \thinspace \omega_{l+1})^{2} \big] \bigg[1 + \operatorname{erf}\big( \theta_{l+1} \thinspace \omega_{l+1} \big) \bigg]^{-1}.
\end{align*}
\end{frame}

%================================= Frame 24 ==================================%
\begin{frame}[label={sec:org003379d}]{ISC for an RBM (\(L=1\) when \(\alpha_1 \gg 1\))}
\begin{itemize}
\item The \emph{first} key result from \cite{bansal2018using} concerns the ISC for RBMs. It said that the ISC saturates as the number of hidden units increases.
\linebreak
\item For the RBM (\(L=1\)), \(\mathcal{C}_{J} (\boldsymbol{N}) = \mathcal{C}_{J} (\alpha_{1})\) reduces to
\[
\underset{\{x, y\}}{\operatorname{saddle}} \thinspace \frac{1}{2} \bigg \{- \sqrt{\alpha_{1}} \thinspace \big(x^{2} + y^{2} \big) + \ln \bigg [1 + \operatorname{erf} \bigg(\frac{- x + \mathrm{i} y}{\sqrt{2 \gamma_{0}}} \bigg) \bigg] \bigg [1 + \operatorname{erf} \bigg(\frac{- x - \mathrm{i} y}{\sqrt{2 \nu_{1}}} \bigg) \bigg ]^{\alpha_{1}} \bigg \}
  \]
\linebreak
\item In our use of Laplace's method we assumed \(N_{0} \to \infty\), so the analogue of this result is the case where \(N_{1} \to \infty\) such that \(N_{1} / N_{0} \equiv \alpha_{1}\) is finite.
\linebreak
\item We numerically solve for the saddle \(\{x^{\text{max}}, y^{\text{max}}\}\) and substitute into the formula for ISC  \(\mathcal{C}_{J} (\alpha_{1})\) to obtain the response of \(\mathcal{C}_{J} (\alpha_{1})\) to \(\alpha_{1}\).
\end{itemize}
\end{frame}

%================================= Frame 25 ==================================%
\begin{frame}[label={sec:org72bd224}]{ISC for an RBM (\(L=1\) when \(\alpha_1 \gg 1\))}
The ISC \(\mathcal{C}_{J} (\alpha_{1})\) saturates to a limiting value as a function of \(\alpha_{1}\). The saturation limit for \(\mathcal{C}_{J} (\alpha_{1}) \approx 0.506\) is lower than \(0.585\).

\begin{figure}[htbp]
  \centering
  \includegraphics[width=.6\linewidth]{~/.local/images/rbm.png}
  \caption{\textbf{ISC} \(\mathcal{C}_{J} (\alpha_{1})\) \textbf{vs.} \(\alpha_{1}\) \textbf{for an RBM with a single hidden layer} \((L=1)\). The saturation of \(\mathcal{C}_{J}\) indicates the limiting ISC value as \(\alpha_{1}\) increases, highlighting the diminishing returns on model capacity.}
  \label{fig:sub1}
\end{figure}
\end{frame}


%================================= Frame 26 ==================================%
\begin{frame}[label={sec:orgb02b4c9}]{ISC for a DBM (\(\alpha_1 \gg 1\) and \(\alpha_{2} \ll 1\))}
\begin{itemize}
\item The \emph{second} key result from \cite{bansal2018using} concerns ISC for DBMs. It said that for small values of \(\alpha_{2}\), \(\mathcal{C}_{J} (\alpha_{1}, \alpha_{2})\) increases linearly with \(\alpha_{2}\).
\linebreak
\item For DBMs with 2 hidden layers (\(L = 2\)), \(\mathcal{C}_J (\boldsymbol{N}) = \mathcal{C}_{J} (\alpha_{1}, \alpha_{2})\) reduces to
 \begin{align*}
 &\underset{\{x_{0}, x_{1}, y_{0}, y_{1}\}}{\operatorname{saddle}} \frac{1}{2} \thinspace \bigg \{- \bigg[ \sqrt{\alpha_{1}} \thinspace \big(x_{0}^{2} + y_{0}^{2} \big) + \sqrt{\alpha_{1} \alpha_{2}} \thinspace \big(x_{1}^{2} + y_{1}^{2} \big) \bigg] + \ln \bigg [1 + \operatorname{erf} \bigg(\frac{- x_{0} + \mathrm{i} y_{0}}{\sqrt{2 \gamma_{0}}} \bigg) \bigg] \\
 &\qquad \qquad + \ln \bigg [1 + \operatorname{erf} \bigg(\frac{- (x_{1} + x_{0}) + \mathrm{i} (y_{1} - y_{0})}{\sqrt{2 (\gamma_{1} + \nu_{1})}} \bigg) \bigg]^{\alpha_{1}} \bigg [1 + \operatorname{erf} \bigg(\frac{- x_{1} - \mathrm{i} y_{1}}{\sqrt{2 \nu_{2}}} \bigg) \bigg]^{\alpha_{2}} \bigg \}
\end{align*}
\linebreak
\item The analogue of this result is that \(N_{0}\), \(N_{1}\), and \(N_{2}\) all approach \(\infty\) but uphold the proportions \(\alpha_{1} \equiv N_{1} / N_{0} > \beta^{-1} = 20 \gg 1\) and \(\alpha_{2} \equiv N_{2} / N_{0} < \beta = 0.05 \ll 1\).
\linebreak
\item We numerically solve for the saddle \(\{x^{\text{max}}_{0}, x^{\text{max}}_{1},  y^{\text{max}}_{0}, y^{\text{max}}_{1}\}\) and substitute into the formula above to obtain the response of \(\mathcal{C}_{J} (\alpha_{1},\,\alpha_{2})\) to \(\alpha_{2}\) for a fixed \(\alpha_{1}\).
\end{itemize}
\end{frame}

%================================= Frame 27 ==================================%
\begin{frame}[label={sec:orgfb9aed2}]{ISC for a DBM (\(\alpha_1 \gg 1\) and \(\alpha_{2} \ll 1\))}
For small values of \(\alpha_{2}\), \(\mathcal{C}_{J} (\alpha_{1}, \alpha_{2})\) increases linearly with \(\alpha_{2}\). In this regime, ISC is saturated for an RBM (\(\alpha_{2} = 0\)) and can only be increased by adding units to a second hidden layer (\(\alpha_{2} > 0\)).

\begin{figure}[htbp]
  \centering
  \includegraphics[width=.6\linewidth]{~/.local/images/dbm.png}
  \caption{\textbf{ISC} \(\mathcal{C}_{J} (\alpha_{1}, \alpha_{2})\) \textbf{vs.} \(\alpha_{2}\). The interplay between \(\alpha_{1}\) and \(\alpha_{2}\) shows how adding units to the second hidden layer can enhance ISC beyond the saturation point of a single-layer RBM. \(\mathcal{C}_{J} (\alpha_{1}, \alpha_{2})\) starts increasing from the saturation limit of \(\mathcal{C}_{J} (\alpha_{1}) \approx 0.506 \) for the RBM.}
  \label{fig:sub2}
\end{figure}
\end{frame}


%================================= Frame 28 ==================================%
\begin{frame}[label={sec:org4b04120}]{Network design under budget \(\alpha_1 (1 + \alpha_2) = c\)}
\begin{itemize}
\item The \emph{third} key result from \cite{bansal2018using} was concerning network design under a budget. It stated that with a budget of \(c N_{0}^{2}\) parameters, multi-layering is \emph{not} recommended when \(c < 1\) whereas for \(c \geq 1\) multi-layering is recommended.
\linebreak
\item If the number of parameters for a DBM with 2 hidden layers (\(L=2\)) is \(p = c N_{0}^2\) for \(c > 0\), then
\[
  p = c N_0^{2} = N_{0} \times N_{1} + N_{1} \times N_{2} = N_{0}^{2} \alpha_{1} (1 + \alpha_{2}) \implies \alpha_1 (1 + \alpha_2) = c.
  \]
\linebreak
\item In other words, the curve \(\alpha_{1} (1 + \alpha_{2}) = c\) separates \emph{realizable} architectures from \emph{non-realizable} ones in the \(\alpha_{1} - \alpha_{2}\) plane.
\linebreak
\item On plotting a heat-map of \(\mathcal{C}_{J} (\alpha_{1}, \alpha_{2})\) over this plane and superimposing the curve \(\alpha_{1} (1 + \alpha_{2}) = c\) associated with a budget, we recover both these conclusions.
\end{itemize}
\end{frame}

%================================= Frame 29 ==================================%
\begin{frame}[label={sec:org58109ed}]{Tight budget (\(c < 1\))}
When \(c < 1\), ISC is maximal for \(\alpha_{1} = c\), \(\alpha_{2} = 0\). \emph{Under a tight budget, there is no gain in model capacity through multilayering.}
\begin{figure}[htbp]
  \centering
  \includegraphics[width=.6\linewidth]{~/.local/images/budget_0.5.png}
  \caption{\textbf{Tight budget} \((c < 1)\): Heatmap of \(\mathcal{C}_{J} (\alpha_{1}, \alpha_{2})\) for realizable networks when \(c = 0.5\). An RBM maximizes model capacity (\(\alpha_{2} = 0\)).}
\end{figure}
\end{frame}


%================================= Frame 30 ==================================%
\begin{frame}[label={sec:orgc836570}]{Flexible budget (\(c \geq 1\))}
When \(c \geq 1\), there exists an optimum \(\alpha_{1}^{\text{max}} \neq 0\), \(\alpha_{2}^{\text{max}} \neq 0\) that maximizes ISC. \emph{Under a flexible budget, there is a gain in model capacity through multilayering.}
\begin{figure}[htbp]
  \centering
  \includegraphics[width=.6\linewidth]{~/.local/images/budget_10.0.png}
  \caption{\textbf{Flexible budget} \((c \geq 1)\): Heatmap of \(\mathcal{C}_{J} (\alpha_{1}, \alpha_{2})\) for realizable networks when \(c = 10.0\). Multi-layering (DBM with 2 hidden layers) maximizes model capacity for some optimal \(\alpha_{1} \neq 0\), \(\alpha_{2} \neq 0\).}
\end{figure}
\end{frame}


%================================= Frame 31 ==================================%
\begin{frame}[label={sec:orgfe08893}]{ISC for a DBM (\(\alpha_{2} \gg 1\))}
Earlier we saw a linear response of \(\mathcal{C}(\alpha_1,\, \alpha_2)\) to \(\alpha_{2}\) for \(\alpha_{1} \gg 1\) and \(\alpha_{2} \ll 1\) (leftmost panel below). We now allow an unbounded increase in \(\alpha_2\) and put \(\mathcal{C}_{J} (\alpha_{1},\, \alpha_{2})\) to test in a previously unexplored regime.
\begin{figure}[htbp]
    \centering
    \includegraphics[width=1.1\linewidth]{~/.local/images/varying-width-dbm.pdf}
    \caption{\textbf{Change in DBM architecture} \(L = 2\) \textbf{as} \(\alpha_1\) \textbf{is fixed and} \(\alpha_2\) \textbf{is increased.}}
    \label{fig:sub2}
  \end{figure}
\end{frame}

%================================= Frame 32 ==================================%
\begin{frame}[label={sec:org7428d27}]{ISC for a DBM (\(\alpha_{2} \gg 1\))}
Focusing on the \(\alpha_1 > \alpha_2 > 1\) regime with an alternative budget scheme \(\alpha_{1} + \alpha_{2} = c\) reveals a contrasting insight. \emph{Premature multi-layering can be sub-optimal: saturating the ISC in an existing layer before adding a new layer tends to yield a larger overall ISC.}
\begin{figure}[htbp]
   \centering
   \includegraphics[width=.6\linewidth]{~/.local/images/dbm2.png}
   \caption{\textbf{ISC} \(\mathcal{C}_{J} (\alpha_{1}, \alpha_{2})\) \textbf{vs.} \(\alpha_{2}\). For \( \alpha_2 \ll 1 \), ISC increases linearly. As \( \alpha_2 \) increases and the network becomes \emph{balanced}, ISC growth slows. For \( \alpha_2 \gg 1 \), ISC grows unconstrained. An alternative budget scheme is shown: a total budget of \( \alpha_1 + \alpha_2 = 10.0 \) is portioned among \(\alpha_1\) and \(\alpha_2\), with the associated ISC marked on each curve with a circular patch.}
\label{fig:sub2}
\end{figure}
\end{frame}

%================================= Frame 33 ==================================%
\begin{frame}[label={sec:orgd4f1dc5}]{Limitations}
\begin{itemize}
\item The \(\alpha_{2} \gg 1\) regime indicates a breakdown of ISC \(\mathcal{C}_{J} (\alpha_{1}, \alpha_{2})\), but not because \(\mathcal{C}_{J} (\alpha_{1}, \alpha_{2}) \leq 1\) is violated.
\linebreak
\item We defined ISC as \(\mathcal{C}_{J} (\boldsymbol{N}) \equiv N_{0}^{-1} \ln \langle \mathcal{N}_{s} \rangle_{J}\) where \(\mathcal{N}_{s}\) represents the modes of \(p(\boldsymbol{\sigma})\) but is by \(N_{0}\). Thus, \(\mathcal{C}_{J} (\alpha_{1}, \alpha_{2}) \geq 1\) is admissible by our definition.
\linebreak
\item The problem is an assumption implicit in our definition: \emph{the number of modes of the marginal distribution over the visible units} 
\[
p(\boldsymbol{\sigma}_{0}) \equiv \sum_{\boldsymbol{\sigma}_{1}} \ldots \sum_{\boldsymbol{\sigma}_{L}} p (\boldsymbol{\sigma})
\]
\emph{is comparable to the number of modes of the joint distribution} \(p (\boldsymbol{\sigma})\).
\end{itemize}

% Speaker notes
\note[item]{The \(\alpha_{2} \gg 1\) regime indicates a breakdown of ISC \(\mathcal{C}_{J} (\alpha_{1}, \alpha_{2})\), but not because \(\mathcal{C}_{J} (\alpha_{1}, \alpha_{2}) \leq 1\) is violated.}
\note[item]{We defined ISC as \(\mathcal{C}_{J} (\boldsymbol{N}) \equiv N_{0}^{-1} \ln \langle \mathcal{N}_{s} \rangle_{J}\) where \(\mathcal{N}_{s}\) represents the modes of \(p(\boldsymbol{\sigma})\) but is by \(N_{0}\). Thus, \(\mathcal{C}_{J} (\alpha_{1}, \alpha_{2}) \geq 1\) is admissible by our definition.}
\note[item]{The problem is an assumption implicit in our definition: \emph{the number of modes of the marginal distribution over the visible units is comparable to the number of modes of the joint distribution} \(p (\boldsymbol{\sigma})\).}
\end{frame}

%================================= Frame 34 ==================================%
\begin{frame}[label={sec:orgf3b83eb}]{Limitations}
\begin{itemize}
\item If a set of visible spins \(\boldsymbol{\sigma}_{0}^{\ast}\) is a mode of the marginal \(p(\boldsymbol{\sigma}_{0})\), there exists a set of hidden spins \((\boldsymbol{\sigma}_{l}^{\ast})_{l=1,\ldots, L}\) such that \(\boldsymbol{\sigma}^{\ast}\) is an inherent structure.
\linebreak
\item For \(L = 1\), there are situations where this relationship is \emph{provably} one-to-one \cite{montufar2015mixture} so that \emph{our assumption is entirely reasonable}.
\linebreak
\item However, for \(L \geq 2\), this relationship is one-to-many: number of modes of \(p (\boldsymbol{\sigma})\) in a \emph{upper bound} for the number of modes of \(p(\boldsymbol{\sigma}_{0})\): its usefulness depends on the severity of this relationship.
\linebreak
\item In the \(\alpha_2 \gg 1\) regime, when \(\mathcal{C}_{J} (\alpha_{1}, \alpha_{2})\) exceeds 1, \emph{our assumption is clearly indefensible}.
\end{itemize}
\end{frame}

%================================= Frame 35 ==================================%
\begin{frame}[label={sec:org6ef07cd}]{Challenges}
\begin{itemize}
\item We cannot redefine \(\mathcal{N}_{s}\) in \(\mathcal{C} (\boldsymbol{N}) \equiv N_{0}^{-1} \ln \langle \mathcal{N}_{s} \rangle_{J}\) and proceed through the same set of steps.
\linebreak
\item To see this, suppose \(\lambda_{il} > 0\) for all \(i\) and \(l\). The following holds \emph{unconditionally}:
\[\sigma_{il} \to -\sigma_{il} \implies \Delta H_{J} > 0 \quad \forall i,~ j.\]
This result is the starting point of the majority of complexity calculations in the literature \cite{bray1981metastable}, \cite{tanaka1980analytic}.
\linebreak
\item Contrast this with the case where \(\lambda_{il} > 0\) for all \(i\) and \(l = 0\), but the values for \(l \neq 0\) are indeterminate. In this case, given \(i\)
\[\sigma_{i0} \to -\sigma_{i0} \centernot\implies \Delta H_J  > 0.\]
Whether \(\Delta H_J > 0\) holds depends on whether \(\sum_{j} \Delta \lambda_{j1} \geq \Delta \lambda_{i0}\).
\end{itemize}
\end{frame}

%================================= Frame 36 ==================================%
\begin{frame}[label={sec:org48e3676},fragile]{Monte Carlo estimation of ISC}
An extensive amount of time has been devoted to studying, improving, and developing efficient code for algorithms to obtain numerical estimates of the \(\mathrm{ISC}\) \cite{schnabel2017dynamic}, \cite{waclaw2008counting}, \cite{schnabel2018distribution}, \cite{schnabel2019distribution}, \cite{schnabel2020counting}.
\linebreak
\begin{enumerate}
\item \alert{\url{https://crates.io/crates/fastset}} which gives a set implementation that beats time-complexity of state of the art set implementations like Google's \texttt{hashbrown::HashSet}. It provides a \texttt{random} method for uniform random sampling from the set, \emph{independent of set size}, with \emph{picosecond latency}.
\linebreak
\item \alert{\url{https://crates.io/crates/signvec}} which extends the capabilities of the traditional containers like Rust's \texttt{std::collection::Vec} with functionality to efficiently track and manipulate elements based on their sign, \emph{independent of the size of the vector}, with \emph{picosecond latency}.
\linebreak
\end{enumerate}
We did not present the details of this investigation and the associated challenges due to the time constraint.
\end{frame}

%================================= Section 5 ==================================%
\section{Future work}
\label{sec:org1bc37d5}
%================================= Frame 37 ====================================%
\begin{frame}[label={sec:org2d79253}]{Future work}
\begin{itemize}
\item Investigating how the architecture of the DBM affects the previously discussed one-to-many relationship between the modes of \(p(\boldsymbol{\sigma}_{0})\) and \(p(\boldsymbol{\sigma})\) \cite{montufar2015mixture}.
\linebreak
\item Improving techniques for sampling the inherent structures of the DBM to make numerical estimation of \(\mathrm{ISC}\) feasible \cite{schnabel2019distribution}.
\linebreak
\item Extending the domain of applicability of the \(\mathrm{ISC}\) beyond Gaussian distributed couplings \cite{ichikawa2022statistical}.
\linebreak
\item Designing DBMs for real-world applications using \(\mathrm{ISC}\) and demonstrating a non-trivial savings in the number of parameters.
\end{itemize}
\end{frame}

%================================= Frame 38 ====================================%
\begin{frame}[allowframebreaks]
\frametitle{References}
\printbibliography
\end{frame}

\end{document}
